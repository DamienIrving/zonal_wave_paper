\section{Introduction}\label{s:introduction}

The relationship between mid-to-upper tropospheric planetary wave activity and surface extremes in the Northern Hemisphere has received a great deal of attention in recent times, as researchers try to better understand the links between the Arctic Amplification and mid-latitude weather \citep[e.g.][]{Cohen2014,Screen2014}. While the meridional temperature gradient has not undergone such dramatic changes in the Southern Hemisphere, this flurry of research activity has served to highlight the deficits in our understanding of the link between planetary wave activity and surface extremes in the Southern Hemisphere. A few recent studies have looked at the relationship with Antarcitc sea ice extent \citep{Raphael2007,Raphael2014} and surface temperatures in West Antarctica \citep{Ding2011,Ding2013} (i.e. topics that much like the Arctic Amplification have made climate change headlines in recent years), but the literature currently lacks a broad, hemispheric perspective of the relationship between planetary wave acitivity and surface extremes. 

In both hemispheres, large-scale topography and continent-ocean heating contrasts provide strong forcing for longitudinally asymmetric planetary scale time-mean motions. Such motions, usually referred to as stationary waves, are especially strong during the winter season. Observations indicate that tropospheric stationary waves tend to have an equivalent barotropic stucture, meaning the wave amplitude generally increases with height, but phase lines tend to be vertical \citep{Holten2013}. 



When superimposed on zonal-mean circulation, such waves produce local regions of enhanced and diminished time mean westerly winds, which strongly influence the development and propagation of transient weather disturbances. Thus they represent essential features of the climatological flow.

Next paragraphs:
\begin{itemize}
\item Explain what we do know about planetary wave activity in the SH: Major recurring patterns are ZW1, ZW3 and to a lesser extent PSA. There are some metrics and climatologies already out there for the ZW1 and ZW3 \citep{Raphael2004,Hobbs2007}.
\item Explain the sea ice and West Antarctica results
\item Explain why, in theory, we would expect planetary waves and extremes to be linked
\end{itemize}
