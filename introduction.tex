\section{Introduction}\label{s:introduction}

The relationship between mid-to-upper tropospheric planetary wave activity and surface extremes in the Northern Hemisphere has received a great deal of attention in recent times, as researchers try to better understand the links between the Arctic Amplification and mid-latitude weather \citep[e.g.][]{Cohen2014,Screen2014}. While the meridional temperature gradient has not undergone such dramatic changes in the Southern Hemisphere, this flurry of research activity has highlighted the deficits in our understanding of the link between planetary wave activity and surface extremes in the Southern Hemisphere. A few recent studies have looked at the relationship with Antarcitc sea ice extent \citep{Raphael2007,Raphael2014} and surface temperatures in West Antarctica \citep{Ding2011,Ding2013} (i.e. topics that much like the Arctic Amplification have made climate change headlines in recent years), but the literature lacks a broad, hemispheric perspective. 

In both hemispheres, large-scale topography and continent-ocean heating contrasts provide strong forcing for longitudinally asymmetric planetary scale time-mean motions. Such motions, usually referred to as stationary or planetary waves, are especially strong during winter and tend to have an equivalent barotropic stucture, meaning the wave amplitude increases with height but phase lines tend to be vertical \citep{Holton2013}. In the context of weather and climate extremes at the surface, these waves are important because they produce local regions of enhanced and diminished time-mean westerly winds, which strongly influence the development and propagation of transient weather disturbances. Persistent (or blocked) weather patterns, for instance, are typically associated with high-amplitude waves in the upper troposphere \citep[e.g.][]{Trenberth1985,Renwick2005}.

It was \citet{vanLoon1972} who first characterised planetary wave activity in the Southern Hemisphere as the super-position of two zonally-oriented, quasi-stationary waveforms; one that repeats itself once around the hemisphere (zonal wave one; ZW1) and another that repeats itself three times (zonal wave three; ZW3). In any Fourier decomposition of the mid-to-upper tropospheric circulation, \citet{vanLoon1972} concluded that waveforms corresponding to other wavenumbers typically only exist to modulate ZW1 and ZW3. Since that landmark study, the ZW1 and ZW3 patterns have been identified as dominant features of the mid-latitude circulation on daily \citep[e.g.][]{Kidson1988}, seasonal \citep[e.g.][]{Mo1985} and interannual \citep[e.g.][]{Karoly1989} timescales, and corresponding metrics and climatologies have been developed \citep{Raphael2004,Hobbs2007}.

While these climatologies reveal many of the basic characteristics of the ZW1 and ZW3 patterns (e.g. their seasonality, spatial pattern, etc), with the exception of the sea ice work of \citet{Raphael2007} and \citet{Raphael2014} subsequent studies have not yet extended these climatologies to look at their influence on key variables like surface temperature and precipitation. One reason for this might be that the ZW1 and ZW3 patterns never really occur in isolation, which makes analyses of just one or the other somewhat problematic. This study will attempt to produce a climatology of total zonal planetary wave activity in the Southern Hemisphere (i.e. ZW1 and ZW3 combined). That climatology will then be used to explore the influence of these waves on surface temperature, precipitation and sea ice.
