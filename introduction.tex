\section{Introduction}\label{s:introduction}

The relationship between mid-to-upper tropospheric planetary wave activity and surface extremes in the Northern Hemisphere has received a great deal of attention in recent times, as researchers try to better understand the links between the Arctic Amplification and mid-latitude weather \citep[e.g.][]{Cohen2014,Screen2014}. While the meridional temperature gradient has not undergone such dramatic changes in the Southern Hemisphere (SH), this flurry of research activity has highlighted the deficits in our understanding of Southern Hemisphere planetary wave activity and its link to surface conditions. 

In both hemispheres, large-scale topography and continent-ocean heating contrasts provide strong forcing for longitudinally asymmetric planetary scale time-mean motions. Such motions, usually referred to as stationary or planetary waves, are especially strong during winter and tend to have an equivalent barotropic structure, meaning the wave amplitude increases with height but phase lines tend to be vertical \citep{Holton2013}. In the context of weather and climate extremes at the surface, these waves are important because they produce local regions of enhanced and diminished time-mean westerly winds, which strongly influence the development and propagation of transient weather disturbances. Persistent (or blocked) weather patterns, for instance, are typically associated with high-amplitude waves in the upper troposphere \citep[e.g.][]{Trenberth1985,Renwick2005}.

It was \citet{vanLoon1972} who first characterized SH planetary wave activity as the super-position of two zonally-oriented, quasi-stationary waveforms of wavenumber one (ZW1) and wavenumber three (ZW3). Based on Fourier decompositions of the mid-to-upper tropospheric circulation, they concluded that the net effect of the other wavenumbers was simply to modulate ZW1 and ZW3. Since that landmark study, the ZW1 and ZW3 patterns have been identified as dominant features of the mid-latitude circulation on daily \citep[e.g.][]{Kidson1988}, seasonal \citep[e.g.][]{Mo1985} and interannual \citep[e.g.][]{Karoly1989} timescales. Corresponding metrics and climatologies have been developed \citep{Raphael2004,Hobbs2007} and their relationship with circulation features including the Amundsen Sea Low \citep{Turner2013} and two prominent quasi-stationary anticyclones in the sub-Antarctic western hemisphere \citep{Hobbs2010} have been investigated.

While these climatologies and investigations reveal many of the basic characteristics of the ZW1 and ZW3 patterns (e.g. their variability and spatial pattern), with the exception of the ZW3 sea ice analyses of \citet{Raphael2007} and \citet{Yuan2008} and the ZW1 sea surface temperature results of \citet{Hobbs2007}, subsequent studies have not yet extended these climatologies to look at their influence on key variables such as surface temperature and precipitation. Related studies \citep[e.g. of Australian rainfall variability;][]{Frederiksen2014} sometimes mention a ZW3-like pattern in passing, but the literature lacks a broad, hemispheric perspective on the link between planetary wave activity and surface extremes. One reason for this might be that the ZW1 and ZW3 patterns never really occur in isolation, which makes analyses of just one or the other somewhat problematic \citep{Hobbs2010}. What is the value of looking at the impacts of just the ZW1 or ZW3 pattern, when the atmospheric state tends to reside in a hybrid of the two? 

In this study we take a new approach to the analysis of SH planetary wave activity. Rather than focus on a specific wavenumber or pattern, we simply consider all times where there is a strong meridional component to the hemispheric flow. We hypothesize that the vast majority of these times are associated with a quasi-stationary, mixed ZW1 / ZW3 planetary wave pattern, and thus strong meridional flow is a useful proxy for planetary wave activity which does not suffer the aforementioned shortcomings of other approaches/metrics (exclusion of important Fourier modes, stationary analysis point, etc). After finding this hypothesis to be true, we consider (a) the climatological characteristics of this planetary wave activity, (b) its impact on surface temperature, precipitation and sea ice and (c) the implications of this new approach for interpreting existing analyses of SH planetary wave activity.

\textit{FIXME: The periodogram therefore becomes evidence in support of our hypothesis (and should be presented earlier on)}