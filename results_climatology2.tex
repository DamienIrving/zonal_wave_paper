Elements of the mixed ZW1 / ZW3 pattern shown in Figure \ref{fig:pwi_spatial_summary} are not unique to days where the PWI is greater than its 90th percentile. As demonstrated in Figure \ref{fig:periodograms} (left panel), the ZW1 component of the flow is relatively insensitive to changes in the strength of the meridional flow. Instead, it appears that the main difference between days of very strong (PWI > 90th pctl) and very weak (PWI < 10th pctl) planetary wave activity is the strength of the ZW3 component. While influential at all times, the ZW3 component is far more pronounced when the meridional flow is strong and therefore dominates the streamfunction anomaly patters (and thus surface impacts) discussed in Section \ref{s:surface_extremes}.

While our focus is on the monthly (i.e. 30 day running mean) timescale, it is interesting to consider whether similar behavior is observed at other timescales. It can be seen from Figure \ref{fig:periodograms} (right panel) that wavenumber three dominates the average periodogram when the running mean applied to the daily 500 hPa meridional wind is greater than 10 days, with wavenumber one becoming progressively more influential as the smoothing increases (Figure \ref{fig:periodograms}). When the same process is repeated using the 500 hPa geopotential height (not shown), the results are very different. The ZW1 dominates at all timescales and except for a slight upswing from wavenumber two to three, the variance explained monotonically decreases for subsequent wavenumbers. This is an important result because \citet{vanLoon1972} analyzed geopotential height data and concluded that ZW1 explains (by an appreciable margin) the largest fraction of the spatial variance in the 500hPa SH circulation (a finding that has been quoted in many subsequent papers). In light of the results presented here and the previous discussion about the fact that $v_k \propto k Z_k$ in Fourier space, it is clear that ZW3 is more dominant than previously thought. 