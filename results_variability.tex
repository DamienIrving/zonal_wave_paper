\subsection{Variability}

A similar composite analysis was used to assess the relationship between the PWI and the major modes of SH climate variability, whereby the composite mean zonal 500 hPa geopotential height anomaly was calculated for positive and negative phases of ENSO and SAM. Positive (El Ni\~{n}o) and negative (La Ni\~{n}a) ENSO events were defined as a Ni\~{n}o 3.4 above 0.5$^{\circ}$C and below -0.5$^{\circ}$C respectively, while SAM events were defined according to the 75th and 25th percentiles of the AOI. The composites for each phase of SAM and ENSO were then calculated by taking the average across all timesteps for which the Ni\~{n}o 3.4 or AOI index was under or over the relevant threshold \textit{and} the PWI exceeded the 90th percentile. 

The circulation anomalies associated positive and negative phases of the SAM composites 