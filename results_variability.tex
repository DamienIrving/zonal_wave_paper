\subsection{Variability}

A similar composite analysis was used to assess the relationship between the PWI and the major modes of SH climate variability. Positive (El Ni\~{n}o) and negative (La Ni\~{n}a) ENSO events were defined as a Ni\~{n}o 3.4 above 0.5$^{\circ}$C and below -0.5$^{\circ}$C respectively, while SAM events were defined according to the 75th and 25th percentiles of the AOI. Composites for each phase of SAM and ENSO were then calculated by taking the average across all timesteps for which the PWI exceeded the 90th percentile \textit{and} the Ni\~{n}o 3.4 or AOI index was greater or less than the relevant threshold. 

The circulation anomalies associated with the SAM composites show that the phase of the zonal planetary wave pattern moves east during positive SAM events and west during negative (Figure \ref{fig:sam_composites}. Planetary wave activity was also more common when the SAM was negative: of the 1291 timesteps where the PWI exceeded the 90th percentile, 499 (39\%) had an AOI value that was less than the 25th percentile as compared to only 168 (13\%) with an AOI greater than the 75th percentile. 

The influence of ENSO on planetary wave activity was far less pronounced. Besides a subtle east (La Ni\~{n}a) / west (El Ni\~{n}o) movement of the anticyclone over the south-east Pacific, no appreciable changes were seen in the phase of the planetary wave pattern (not shown) and planetary wave activity was only slightly more common during El Ni\~{n}o conditions (283 timesteps to 177).      