Southern Hemisphere mid-to-upper tropospheric planetary wave activity is characterized by the superposition of two zonally-oriented, quasi-stationary waveforms: zonal wavenumber one (ZW1) and zonal wavenumber three (ZW3). Previous studies have tended to consider these waveforms in isolation and with the exception of sea ice, little is known about their impact on surface extremes. We take a novel approach to quantifying the combined influence of ZW1 and ZW3, using the strength of the hemispheric meridional flow as a proxy for zonal wave activity. Our methodology adapts the wave envelope construct routinely used in the identification of synoptic-scale Rossby wave packets and improves on existing ZW1 and ZW3 metrics by allowing for variations in both wave phase and amplitude. While ZW1 and ZW3 are prominent features of the climatological circulation, the defining feature of highly meridional hemispheric states is the enhancement of the ZW3 component. Composites of the mean surface conditions during these highly meridional, ZW3-like anomalous states (i.e. months of strong planetary wave activity) reveal large sea ice concentration anomalies over the Amundsen and Bellingshausen Seas during autumn and along much of the East Antarctic coastline throughout most of the year. Strong planetary wave activity is also associated with large precipitation anomalies in regions where orographic precipitation dominates (e.g. New Zealand, Patagonia, coastal Antarctica), wet and cool conditions over eastern and southern Australia during spring and anomalously warm temperatures over much of the Antarctic continent. The latter has potentially important implications for the interpretation of recent warming over West Antarctica and the Antarctic Peninsula.
