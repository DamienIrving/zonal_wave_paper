Southern Hemisphere mid-to-upper tropospheric planetary wave activity is characterized by the superposition of two zonally-oriented, quasi-stationary waveforms: zonal wavenumber one (ZW1) and zonal wavenumber three (ZW3). Previous studies have tended to consider these waveforms in isolation and with the exception of sea ice, little is known about their impact on surface extremes. 

Borrowing from recent advances in the automated identification of Rossby wave packets, this study makes use of a signal processing technique based on the Hilbert transform to define a new metric of total zonal wave activity. By capturing the envelope of the combined ZW1 and ZW3 waveform, it improves on existing metrics by allowing for variations in both wave phase and amplitude.

Composites of the mean surface conditions for months of strong planetary wave activity reveal highly anomalous sea ice concentrations over the Amundsen/Bellingshausen Seas during autumn and along much of the East Antarctic coastline throughout most of the year. Strong planetary wave activity is also associated with highly anomalous precipitation in regions where orographic precipitation dominates (e.g. New Zealand, Chile, coastal Antarctica), wet and cool conditions over eastern and southern Australia during spring and anomalously warm temperatures over much of the Antarctic continent. The latter has potentially important implications for the interpretation of recent warming over West Antarctica and the Antarctic Peninsula.
