Southern Hemisphere mid-to-upper tropospheric planetary wave activity is characterised by the superposition of two zonally-oriented, quasi-stationary waveforms; one that repeats itself once around the hemisphere (zonal wave one; ZW1) and another that repeats itself three times (zonal wave three; ZW3). Previous studies have tended to consider these waveforms in isolation and with the exception of sea-ice, little is known about their impact on surface extremes. 

Borrowing from recent advances in the automated identification of Rossby wave packets, this study makes use of a signal processing technique known as a Hilbert Transform to define a new metric of total planetary wave activity. It captures both the ZW1 and ZW3 patterns and improves on existing metrics by allowing for variations in both wave phase and amplitude.