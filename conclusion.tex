\Section{Conclusion}

This paper outlines a novel approach to quantifying the climatological characteristics of quasi-stationary planetary waves. Borrowing from recent advances in the automated identification of synoptic-scale Rossby wave packets, the approach improves on existing methods by allowing for variations in both wave phase and amplitude. In order to demonstrate its utility, the new approach has been used to characterize SH zonal wave activity and its influence on regional climate variability. Our analysis reveals that while both ZW1 and ZW3 are prominent features of the climatological SH circulation, the defining feature of highly meridional hemispheric states is an enhancement of the ZW3 component. These enhanced ZW3 states are associated with large sea ice anomalies over the Amundsen and Bellingshausen Seas and along much of the East Antarctic coastline, large precipitation anomalies in regions of significant topography and anomalously warm temperatures over much of the Antarctic continent. 
