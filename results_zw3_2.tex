Given the clear dominance of wavenumber three from around 10 to 90 days of smoothing, it is interesting to consider the merits of the PWI as an alternative metric for the ZW3 pattern on monthly and seasonal timescales. A direct comparison of the PWI and the ZW3 index of \citet{Raphael2004} for 30-day running mean data shows a reasonably high level of agreement (as evidenced by the linear regression line in Figure \ref{fig:metric_vs_zw3}), however there are some important discrepancies. In particular, the colour of the dots in Figure \ref{fig:metric_vs_zw3}, which represent the phase of the wavenumber three component of the Fourier Transform, are not randomly distributed. Whenever the phase of the wavenumber three component of the flow does not match up with the location of the three grid points used to calculate the ZW3 index (indicated by the dark red and dark blue colours), a low value is recorded for the ZW3 index. This is particularly problematic for outlying dots in the bottom right hand quadrant, where the PWI (and hence in most cases the amplitude of the wavenumber three component of the flow) is actually quite large.      

\textit{FIXME: In this section I might want to briefly point out that wave 3 is most dominant in meridional wind data while wave 1 is in geopotential height data (I could include the geopotential height periodogram in an appendix)? This is important because \citet{vanLoon1972} looked at geopotential height data and concluded that ZW1 accounts for 90\% of the Southern Hemisphere circulation's spatial variation, and lots of authors quote that fact in their introductions. IAN: At a previous meeting you had a nice mathematical explanation for why ZW3 variability is much higher in meridional wind data...? }