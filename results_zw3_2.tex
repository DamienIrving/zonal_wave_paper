\subsubsection{ZW3 index}

Given the clear dominance of wavenumber three from around 10 to 90 days of smoothing, it is not surprising that the ZW3 index \citep{Raphael2004} shows a reasonably high level of agreement with the PWI (Figure \ref{fig:metric_vs_zw3}). Having said that, it is important to note that the colors of the dots in Figure \ref{fig:metric_vs_zw3} --- which represent the phase of the wavenumber three component of the Fourier Transform --- are not randomly distributed. Whenever the phase of the wavenumber three component of the flow does not match up with the location of the three grid points used to calculate the ZW3 index (days of phase inconsistency are indicated by the dark red and dark blue colors), a low value is recorded for the ZW3 index. This is particularly problematic for outlying dots in the bottom right hand quadrant, where the PWI (and hence in most cases the amplitude of the wavenumber three component of the flow) is actually quite large. This finding suggests that there are an appreciable number of phase inconsistent days where the wavenumber three component of the flow is large and hence the stationarity of the ZW3 index might be an important shortcoming. 