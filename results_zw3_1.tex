\subsection{Comparison with alternative indices}

\subsubsection{Fourier Transform}

The dominance of the ZW1 and ZW3 patterns on SH planetary wave activity is further emphasized when considering the individual Fourier components calculated during the Hilbert Transform. Wavenumber three dominates the average periodogram when the running mean applied to the daily 500 hPa meridional wind data is greater than 10 days, with wavenumber one becoming progressively more influential as the smoothing increases (Figure \ref{fig:fourier_spectrum}).


\textit{FIXME: In this section I might want to briefly point out that wave 3 is most dominant in meridional wind data while wave 1 is in geopotential height data (I could include the geopotential height periodogram in an appendix)? This is important because \citet{vanLoon1972} looked at geopotential height data and concluded that ZW1 accounts for 90\% of the Southern Hemisphere circulation's spatial variation, and lots of authors quote that fact in their introductions. IAN: At a previous meeting you had a nice mathematical explanation for why ZW3 variability is much higher in meridional wind data...? }