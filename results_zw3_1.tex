\subsection{Alternative indices and approaches}

\subsubsection{Geopotential height}\label{s:geopotential_height}

The strong ZW3 signature in the preceding composites is further emphasized when considering the individual Fourier components calculated during the Hilbert Transform. Wavenumber three dominates the average periodogram when the running mean applied to the daily 500 hPa meridional wind is greater than 10 days, with wavenumber one becoming progressively more influential as the smoothing increases (Figure \ref{fig:fourier_spectrum}). When the same process is repeated using the 500 hPa geopotential height (not shown), very different results are obtained. The ZW1 dominates at all timescales and the variance explained monotonically decreases for subsequent wavenumbers. This is an important result because \citet{vanLoon1972} analyzed geopotential height data and concluded that ZW1 accounts for 90\% of the spatial variance in the SH circulation (a statistic that has been quoted in many subsequent papers). 