\subsection{Alternative indices and approaches}

In addition to providing information on the climatological characteristics of SH planetary wave activity and its influence on surface extremes, our analysis can also provide valuable insights into a number of related indices and approaches to analyzing the SH circulation.

\subsubsection{Geopotential height}\label{s:geopotential_height}

The strong ZW3 signature shown in the preceding composites is further emphasized when considering the individual Fourier components calculated during the Hilbert Transform. Wavenumber three dominates the average periodogram when the running mean applied to the daily 500 hPa meridional wind is greater than 10 days, with wavenumber one becoming progressively more influential as the smoothing increases (Figure \ref{fig:fourier_spectrum}). When the same process is repeated using the 500 hPa geopotential height (not shown), the results are very different. The ZW1 dominates at all timescales and except for a slight upswing from wavenumber two to three, the variance explained monotonically decreases for subsequent wavenumbers. This is an important result because \citet{vanLoon1972} analyzed geopotential height data and concluded that ZW1 explains (by an appreciable margin) the largest fraction of the spatial variance in the 500hPa SH circulation (a finding that has been quoted in many subsequent papers). In light of the results presented here and the previous discussion about the fact that $v_k \propto k Z_k$ in Fourier space, it is clear that ZW3 is more dominant than previously thought. 