\subsection{Surface extremes}

Two approaches were taken to assessing the influence of planetary wave activity on surface extremes. For the first, the composite mean of a variable of interest (surface air temperture anomaly, precipitation anomaly or sea ice fraction anomaly) was calculated for all timesteps where the planetary wave index exceeded the 90th percentile. In other words, what is the average temperature (or precipitation or sea ice fraction) anomaly when there is strong planetary wave activity? To check the robustness of these results the reverse approach was also applied, whereby the average planetary wave index was calculated at each grid point  

\subsubsection{Surface air temperature}

MAM and JJA heat throughout all of Antarctica. Particular focus 

%mention/check that the result is no different using linearly detrended temperature data