\subsection{Temporal characteristics}

Consistent with previous studies \citep{vanLoon1984,Mo1985}, the composite mean 500 hPa zonal streamfunction anomaly pattern for days exceeding the 90th percentile (Figure \ref{fig:pwi_spatial_summary}) migrates zonally by approximately 15$^{\circ}$ from its most easterly location during summer to its most westerly disposition during winter (notwithstanding the fact that the pattern breaks down from around 240-330$^{\circ}$E during summer). It has a slightly larger amplitude during the winter months and the frequency of strong planetary wave activity is also far more pronounced at that time of the year (Figure \ref{fig:annual_distribution}). The are no statistically significant linear trends in planetary wave activity for any season, however 1980 was associated with particularly strong wave activity.

While our focus is on the monthly (i.e. 30 day running mean) timescale, it is interesting to consider whether similar behavior is observed at other timescales. It can be seen from Figure \ref{fig:periodograms} (right panel) that wavenumber three dominates the average periodogram when the running mean applied to the daily 500 hPa meridional wind is greater than 10 days, with wavenumber one becoming progressively more influential as the smoothing increases (Figure \ref{fig:periodograms}). When the same process is repeated using the 500 hPa geopotential height (not shown), the results are very different. The ZW1 dominates at all timescales and except for a slight upswing from wavenumber two to three, the variance explained monotonically decreases for subsequent wavenumbers. This is an important result because \citet{vanLoon1972} analyzed geopotential height data and concluded that ZW1 explains (by an appreciable margin) the largest fraction of the spatial variance in the 500hPa SH circulation (a finding that has been quoted in many subsequent papers). In light of the results presented here and the previous discussion about the fact that $v_k \propto k Z_k$ in Fourier space, it is clear that ZW3 is more dominant than previously thought, particularly when there is a strong meridional component to the hemispheric flow. 


