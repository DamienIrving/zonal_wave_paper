Given the dominance of the ZW3, it is not surprising that the ZW3 index of \citet{Raphael2004} shows a reasonably high level of agreement with the PWI (Figure \ref{fig:metric_vs_zw3}). Having said that, it is important to note that the colors of the dots in Figure \ref{fig:metric_vs_zw3} --- which represent the phase of the wavenumber three component of the Fourier Transform --- are not randomly distributed. Whenever the phase of the wavenumber three component of the flow does not match up with the location of the three grid points used to calculate the ZW3 index (indicated by the dark red and dark blue colors), a low value is recorded for the ZW3 index. The outlying dots in the bottom right hand quadrant are particularly noteworthy, as in these cases the PWI (and hence in most cases the amplitude of the wavenumber three component of the flow) is actually quite large. The failure of the ZW3 index to capture these out of phase patterns means that composite analyses based on that index may overstate the stationarity (and hence impacts) of the ZW3 component of the flow.