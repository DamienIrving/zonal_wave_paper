\section{Methodology}

In analysing planetary wave activity in the Southern Hemisphere (i.e. the ZW1 and/or ZW3 patterns), previous studies have tended to define metrics based around either grid point values or a Fourier decomposition. With respect to the former, \citep[e.g.][]{Raphael2004} define a ZW3 index that is essentially the average geopontial height anomaly across three key points (the approximate location of the three geopotential height maxima around the hemisphere). An attractive feature of this metric is its simplicity (it has has been applied in subsequent studies of Antarctic sea ice variability \citep{Raphael2007,Raphael2014}), however its stationary nature means that it cannot fully capture the subtle (but not insignificant at around 15 degrees of longitude) seasonal migration in the phase of the ZW3.


\citep[e.g.][]{Raphael2004,Raphael2014} or Fourier transform \citep[e.g.][]{Hobbs2007,Hobbs2010}. With respect to the former, \cite



With respect to the latter, the Fourier Transform is used to express the upper tropospheric geopotential height in the frequency domain as opposed to the spatial domain. The output of the transform is essentially a magnitude and phase coefficient for each frequency/wavenumber. 