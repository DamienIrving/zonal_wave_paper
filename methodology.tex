\section{Methodology}

In analysing planetary wave activity in the Southern Hemisphere (i.e. the ZW1 and/or ZW3 patterns), previous studies have tended to define metrics based on either grid point values or Fourier decomposition. With respect to the former, \citet{Raphael2004} define a ZW3 index that is essentially the average geopotential height anomaly across three key points (the approximate location of the three geopotential height maxima around the hemisphere). An attractive feature of this metric is its simplicity (it has been applied in subsequent studies of Antarctic sea ice variability \citep{Raphael2007,Raphael2014}), however its stationary nature means that it cannot fully capture the subtle (but not insignificant at around 15 degrees of longitude) seasonal migration in the phase of the ZW3, or the occurence of patterns whose phase is not close to average.

\citet{Hobbs2007} and \citet{Hobbs2010} take an alternative approach, using a Fourier Transform to express the upper tropospheric geopotential height in the frequency domain as opposed to the spatial domain. The output of a Fourier Transform is essentially a magnitude and phase coefficient for each frequency/wavenumber, so those studies simply picked out the coefficients corresponding to the ZW1 (and ZW3 in the case of \citet{Hobbs2010}). While this is an improvement on the grid point method in the sense that the phase is allowed to vary, a shortcoming is that the result is a constant amplitude wave over the entire domain. In many cases ZW3-like variability is only prevalent over part of the hemisphere, and the other Fourier compoents are required to modulate the amplitude accordingly (as alluded to in the seminal work of \citet{vanLoon1972}).

None of the aforementioned studies attempted to combine their ZW1 and ZW3 metrics to get a measure of total planetary wave activity, so for an example of this we must turn to the Northern Hemisphere. In analysing the relationship between planetary wave activity and regional weather extremes, \citet{Screen2014} calculated the magnitude coefficient (from a Fourier Transform of the 500hPa geopotential height) for a range of wavenumbers of interest, and then simply counted the number of positive and negative anomalies. While this may be an appropriate appraoch for the Northern Hemisphere, it too fails to account for that fact that some of the waveforms in a Fourier Transform simply exist to modulate others (i.e. it may not be apporpriate to count these).  

Our solution to this problem 
