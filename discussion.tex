\section{Discussion and conclusions}

A new approach has been developed to describe the climatological characteristics of SH planetary wave activity and its influence on surface extremes. It uses the strength of the hemispheric meridional flow as a proxy for planetary wave activity and quantifies this property using a wave envelope methodology that has recently been applied in identifying synoptic scale Rossby wave packets. In interpreting the results derived from this new approach, it is important to consider what is meant by the phrase `planetary wave activity'. It is evident from our analysis (Figure \ref{fig:periodograms}, left panel) that the ZW1 and ZW3 patterns are a prominent feature of the SH circulation even when the hemispheric meridional flow is weak. As the meridional flow gets stronger the ZW3 component becomes increasingly prominent, while the ZW1 component remains relatively unchanged. This means the average anomalous flow associated with a highly meridional hemispheric state clearly resembles a ZW3 pattern (Figures \ref{fig:tas_composite}, \ref{fig:pr_composite} and \ref{fig:sic_composite}). It is this highly meridional and anomalous ZW3 circulation that the PWI captures and thus we refer to as planetary wave activity, as opposed to the mixed ZW1 / ZW3 pattern that is essentially present at all times.  

Our climatology of planetary wave activity confirms previous results regarding the seasonality of the zonal waves (peak activity in winter, seasonal migration of the zonal location/phase), and also identifies a large sector of the western hemisphere (120-30$^{\circ}$W) where the mean wave activity breaks down during summer. In contrast to the results presented here, previous studies have suggested a link between stationary wave activity and ENSO \citep[e.g.][]{Trenberth1980,Raphael2003,Hobbs2007}. Given the hemispheric nature of the PWI (i.e. it responds most strongly to coordinated, hemispheric patterns of meridional flow) it is perhaps not surprising that we found no link with ENSO, given that teleconnections between ENSO and the high southern latitudes tend to be localized around the southeast Pacific \citep{Turner2004}. The identified east/west migration of the mean planetary wave pattern with positive/negative phases of the SAM possibly ties in with the zonally asymmetric properties of the SAM, however a detailed analysis of this relationship was beyond the scope of thus study. An analysis of the relationship between the PWI and variables such as ocean temperature and stratospheric ozone was also beyond the scope of this paper, but such an analysis would be warranted in considering the possible predictability of zonal wave variability.

With respect to the link between planetary wave activity and surface extremes, most relevant investigations focus on sea ice. The recent study of \citet{Raphael2014} takes a new approach to assessing the influence of the atmospheric circulation, focusing on the ice advance (approximately March-August) and retreat (September-February) seasons for five distinct regions of sea ice variability around Antarctica. Examination of the spatial pattern of correlation between sea ice extent and 500 hPa geopotential height for each season/region suggests that the ZW3 pattern is the primary driver of sea ice variability in the Weddell and Amundsen/Bellingshausen Seas during the advance season. Our results tend to support this finding, particularly during the early part (MAM) of the advance season. In contrast, the strong association identified between the PWI and sea ice coverage just to the north of George V Land, Ad{\'e}lie Land and the Sabrina Coast in East Antarctica does not seem to be in agreement with the results of \citet{Raphael2014}, who found the SAM to be the major driver in that region for both the advance and retreat seasons. 

For the King Haakon VII Sea (10$^{\circ}$W-70$^{\circ}$E), \citet{Raphael2014} were unable to identify an obvious atmospheric driver. Our results suggest that zonal planetary wave activity may play an important role there, since the correlation patterns identified by \citet{Raphael2014} bear some resemblance to the mean planetary wave patterns identified in this study. The reason the resemblance is not stronger may be due to the fact that in some seasons and regions the association between the PWI and sea ice coverage is unidirectional. In MAM, for instance, PWI values greater than the 90th percentile are associated with anomalously low sea ice concentrations, while values less than the 10th percentile are associated with near average (as opposed to anomalously high) concentrations (not shown).

For regions of the Southern Hemisphere where orographic precipitation dominates (e.g. New Zealand, Chile, coastal Antarctica), zonal planetary wave activity stands out as an important driver of precipitation variability (\textit{FIXME: Look at articles on rainfall variability in these places and see if zonal waves are mentioned}). It also appears to be associated with wet and cool conditions over eastern and southern Australia during spring. \textit{FIXME: Comment on the existing Australian rainfall variability literature. The overview that everyone cites is \citet{Risbey2009}, which clearly shows that ENSO is the main driver of rainfall variability in spring (perhaps all we're seeing in our precipitation composites is the relationship between the PWI and ENSO, although we found no relationship between those two annually - could check seasonally), while a recent study by \citet{Frederiksen2014} actually mentions ZW3 but mainly with respect to winter rainfall variability in Australia.}

Aside from Australia, the only other Southern Hemisphere locations for which planetary wave activity is associated with large cold temperature anomalies is the Weddell Sea (in autumn and winter) and Ross Sea (spring). In the main, the enhanced meridional flow brings warm air poleward and large positive temperature anomalies are seen throughout most of Antarctica, particularly during autumn and winter. Zonal wave activity is scarcely mentioned in overviews and analyses of Antarctic temperature variability \citep[e.g.][]{Russell2010,SchneiderOkumura2012,Yu2012}, so our results highlight this as a potentially important oversight. The link between zonal planetary wave activity and West Antarctic temperature variability is particularly interesting, given the large positive temperature trends observed in that region over recent decades \citep[e.g.][]{Bromwich2013}. Among a number of other seasons and regions, the winter trends over the interior of West Antarctica \citep{Ding2011} and the spring trends over the western aspect of the Antarctic Peninsula \citep{Ding2013} have been linked to the Pacific-South American (PSA) pattern, which is the most prominent non-zonal planetary wave pattern in the Southern Hemisphere \citep[e.g.][]{Mo2001}. Temperature variability in these seasons/regions was shown here to be strongly associated with the PWI, so future work will attempt to disentangle the influence of the PSA and zonal wave patterns in the region, so as to better understand the role of both in recent trends.    

In characterizing the PSA pattern, this future work will further exploit the utility of the wave envelope by considering a more restricted wavenumber range and situations where the mean meridional flow is non-zero. It seems likely that the wave envelope would also be a useful tool in future investigations of Northern Hemisphere planetary wave activity, particularly in relation to identifying possible changes related to the Arctic Amplification.  


