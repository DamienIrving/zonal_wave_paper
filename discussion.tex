\section{Discussion and conclusions}

A novel method for identifying quasi-stationary planetary wave activity has been developed and applied to the problem of characterizing the SH zonal waves and their influence on regional climate variability. The method adapts the wave envelope construct traditionally used in the identification of synoptic-scale Rossby wave packets and improves on existing methods by allowing for variations in both wave phase and amplitude. The zonal wave analysis reveals that while both ZW1 and ZW3 are prominent features of the climatological SH circulation, the defining feature of highly meridional hemispheric states is an enhancement of the ZW3 component. These enhanced ZW3 states are associated with large sea ice anomalies over the Amundsen and Bellingshausen Seas and along much of the East Antarctic coastline, large precipitation anomalies in regions of significant topography and anomalously warm temperatures over much of the Antarctic continent. 

In interpreting the results of our zonal wave analysis, it is important to clearly define what is meant by the phrase `quasi-stationary planetary wave activity' in the context of this study. It was evident from our analysis of the monthly timescale (30-day running mean) meridional wind  (Figure \ref{fig:periodograms}a) that the ZW1 and ZW3 patterns are a prominent feature of the SH circulation even when the hemispheric meridional flow is weak. As the meridional flow gets stronger the ZW3 component becomes increasingly prominent, while the ZW1 component remains relatively unchanged. This means the average anomalous flow associated with a highly meridional hemispheric state clearly resembles a ZW3 pattern (Figures \ref{fig:tas_composite}, \ref{fig:pr_composite} and \ref{fig:sic_composite}). It is this highly meridional and anomalous ZW3 circulation that high values of our PWI captures and thus we refer to as planetary wave activity, as opposed to the mixed ZW1 / ZW3 pattern that is essentially present at all times.  

Our climatology of planetary wave activity confirms previous results regarding the seasonality of the zonal waves (peak activity in winter, seasonal migration of the zonal location/phase), and also identifies a large sector of the western hemisphere (120-30$^{\circ}$W) where the mean wave activity breaks down during summer. In contrast to the results presented here, previous studies have suggested a link between planetary wave activity and ENSO \citep[e.g.][]{Trenberth1980,Raphael2003,Hobbs2007}. Given the hemispheric nature of the PWI (i.e. it responds most strongly to coordinated, hemispheric patterns of meridional flow) it is perhaps not surprising that we found no strong link with ENSO, given that teleconnections between ENSO and the high southern latitudes tend to be localized around the southeast Pacific \citep{Simmonds1995,Turner2004}. While this result is not good news for the predictability of planetary wave activity, its increased frequency during negative SAM events offers some hope. The identified east/west migration of the mean planetary wave pattern with positive/negative phases of the SAM possibly ties in with the zonally asymmetric properties of the SAM \citep[e.g.][]{Kidson1988,Kidston2009}, however a detailed analysis of this relationship was beyond the scope of this study.

With respect to the link between planetary wave activity and regional climate variability, most relevant investigations have focused on sea ice. The recent study of \citet{Raphael2014} takes a new approach to assessing the influence of the atmospheric circulation, focusing on the ice advance (approximately March-August) and retreat (September-February) seasons for five distinct regions of sea ice variability around Antarctica. Their examination of the spatial pattern of correlation between sea ice extent and 500 hPa geopotential height for each season/region suggests that the ZW3 pattern is the primary driver of sea ice variability in the Weddell and Amundsen/Bellingshausen Seas during the advance season. Our results tend to support this finding, particularly during the early part (MAM) of the advance season. In contrast, the strong association identified between the PWI and sea ice coverage just to the north of George V Land, Ad{\'e}lie Land and the Sabrina Coast in East Antarctica does not seem to be in agreement with the results of \citet{Raphael2014}, who found the SAM to be the major driver in that region for both the advance and retreat seasons.

For the King Haakon VII Sea (10$^{\circ}$W-70$^{\circ}$E), \citet{Raphael2014} were unable to identify an obvious atmospheric driver. Our results suggest that planetary wave activity may play an important role there, since the correlation patterns identified by \citet{Raphael2014} bear some resemblance to the mean planetary wave patterns identified in this study. The reason the resemblance is not stronger may be due to the fact that the association between the PWI and sea ice coverage appears to be unidirectional in that region. In MAM, JJA and SON, PWI values greater than the 90th percentile are associated with anomalously low sea ice concentrations, while values less than the 10th percentile are associated with near average (as opposed to anomalously high) concentrations (not shown). Of course, any discussion of the atmospheric drivers of sea ice variability is complicated by the relationships between those drivers. For instance, the SAM and ENSO show many similarities in their influence on sea ice. It is unclear whether this is because they operate together in their response mechanism, or if the similarity is due to a preferred hemispheric planetary wave response \citep[e.g.][]{Pezza2012}. 

In contrast to the sea ice literature, planetary wave activity is scarcely mentioned in relation to SH precipitation variability, even in the regions of significant topography so clearly identified in this study. Instead, analyses of precipitation variability over New Zealand, Patagonia and eastern Australia tend to focus on the SAM and ENSO, with the former generally becoming increasingly influential at higher latitudes \citep[e.g.][]{Ummenhofer2007,Aravena2009,Kidston2009,Risbey2009,Garreaud2013,Jiang2013}. Such analyses may inadvertently capture some of the zonal wave influence due to its similarity with the zonally asymmetric features of the SAM, however \citet{Garreaud2013} do note that winter precipitation anomalies over Patagonia are dominated by a wavenumber three mode rather than a more zonally symmetric SAM pattern.

Planetary wave activity also receives scant attention in overviews and analyses of Antarctic temperature variability \citep[e.g.][]{Russell2010,SchneiderOkumura2012,Yu2012}. In the main, we find that the enhanced meridional flow associated with planetary wave activity brings warm air poleward and thus large positive temperature anomalies are seen throughout most of Antarctica, particularly during autumn and winter. The link between planetary wave activity and West Antarctic temperature variability is particularly interesting, given the large positive temperature trends observed in that region over recent decades \citep[e.g.][]{Bromwich2013}. The winter trends over the interior of West Antarctica \citep{Ding2011} and the spring trends over the western aspect of the Antarctic Peninsula \citep{Ding2013} have been linked to the Pacific-South American (PSA) pattern, which is the most prominent non-zonal planetary wave pattern in the Southern Hemisphere \citep[e.g.][]{Mo2001}. Temperature variability in these seasons/regions was shown here to be strongly associated with the PWI, so future work will attempt to disentangle the influence of the PSA and zonal wave patterns in the region, so as to better understand the role of both in recent trends.    

In characterizing the PSA pattern, this future work will further exploit the utility of the wave envelope by considering a more restricted wavenumber range and situations where the mean meridional flow is non-zero. In fact, the demonstrated utility and flexibility of the wave envelope suggests that it might also be a useful tool in similar NH investigations. The debate on the link between the Arctic Amplification and planetary wave activity is far from settled, so the application of the wave envelope has the potential to yield important new insights.    

    