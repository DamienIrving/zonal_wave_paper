\section{Discussion}

The discussion will contain a recap of the methodology, followed by a discussion of my results as they relate to the literature:
\begin{itemize}
\item General climatological characteristics
\begin{itemize}
\item Alternative view of Hobbs (mention summer breakdown)
\item Not strongly linked to SAM or ENSO
\item Wave 1 is dominant if you look at geopotential height (as previous authors have), but wave 3 is for the meridional wind.
\end{itemize}
\item Sea ice (contrast with Raphael)
\item Rainfall reductions for mountainous regions in Southern Hemisphere
\item Cold and wet in southern/eastern Australia during spring
\item Temperatures elsewhere in Antarctica
\item Future work disentangling PSA and West Antarctic trends
\end{itemize}

A new metric of planetary wave activity has been defined that captures the combined influence of the two major zonal waves in the Southern Hemisphere mid-to-upper tropospheric flow. It improves on existing metrics of ZW1 and ZW3 activity by allowing for variations in both wave phase and amplitude. The issue of amplitude variation is particularly important, as \citet{Hobbs2010} recently found that it is somewhat inappropriate to represent individual zonal wave "events" as a constant amplitude ZW1 or ZW3 pattern. Their solution to this problem was to simply consider the two Pacific anticyclones (near Chile and NZ) in isolation, however we find that our Planetary Wave Index 

The climatology of planetary wave activity dervied from the metric confirms previous results regarding the seasonality of zonal wave activity (peak activity in winter, seasonal migration in the zonal location/phase), and also indentifies a large sector of the western hemisphere () where the mean wave activity breaks down during summer. 







