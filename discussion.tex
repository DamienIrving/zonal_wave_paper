\section{Discussion and conclusions}

The discussion will contain a recap of the methodology, followed by a discussion of my results as they relate to the literature:
\begin{itemize}
\item General climatological characteristics
\begin{itemize}
\item Alternative view of Hobbs (mention summer breakdown)
\item Not strongly linked to SAM or ENSO
\end{itemize}
\item Sea ice (contrast with Raphael)
\item Rainfall reductions for mountainous regions in Southern Hemisphere
\item Cold and wet in southern/eastern Australia during spring
\item Temperatures elsewhere in Antarctica
\item Future work disentangling PSA and West Antarctic trends
\end{itemize}

A new metric of planetary wave activity has been defined that captures the combined influence of the two major zonal waves in the Southern Hemisphere mid-to-upper tropospheric flow. It improves on existing metrics of ZW1 and ZW3 activity by allowing for variations in both wave phase and amplitude. \citet{Hobbs2010} recently suggested that it is somewhat inappropriate to represent individual zonal wave "events" as constant amplitude ZW1 or ZW3 patterns. Their solution to this problem was to simply consider the two Pacific anticyclones asssociated with the ZW3 pattern in isolation, however we find that by allowing the amplitude to vary (via the use of a Hilbert Transform) it is both possible and appropriate to consider hemispheric patterns of zonal wave activity.  

The climatology dervied from the PWI confirms previous results regarding the seasonality of zonal wave activity (peak activity in winter, seasonal migration in the zonal location/phase), and also indentifies a large sector of the western hemisphere (120-30$^{\circ}$W) where the mean wave activity breaks down during summer. FIXME: Comment on variability of the metric and links to ENSO and SAM.

With respect to the link between zonal wave activity and surface extremes, the existing literature focuses almost exclusively on sea ice. The latest study from \citet{Raphael2014} takes a new approach to asessing the influence of the atmospheric circulation, focusing on the ice advance (approximately March-August) and retreat (September-February) seasons for five distinct regions of sea ice variability around Antarctica. From looking at the spatial pattern of correlation between the sea ice extent and 500 hPa geopotential height for each season/region, they suggest that the ZW3 pattern is the primary driver of sea ice variability in the Weddell and Amundsen/Bellingshausen Seas during the advance season. Our results tend to support this finding, particularly during the early part (MAM) of the advance season. In contrast, the strong assocation identified between the PWI and sea ice coverage just to the north of George V Land, Ad{\'e}lie Land and the Sabrina Coast in East Antarctica was not supported by \citet{Raphael2014}, who found the SAM to be the major driver in that region for both the advance and retreat seasons. 

Interestingly, \citet{Raphael2014} were unable to identify an obvious atmospheric driver for the King Hakon VII region (10$^{\circ}$W-70$^{\circ}$E) \citet{Raphael2014}. Our results suggest that zonal planetary wave activity may play an important role.  


- They also may provide a possible solution for King Hakon, which Raphael2014 did not nominate a driver. For the advance region in particular. Both patterns look a little zonal wave-like, particularly when you consider that my sea ice stuff only looks at one side of the coin: strong planetary wave activity (which produces a reduction of sea ice in the region). A lack of planetary wave activity does not always produce the opposite results, which might explain why the atmospheric patterns aren't a perfect match.








