\section{Discussion and conclusions}

A new metric of planetary wave activity has been defined that captures the combined influence of the two major zonal waves in the Southern Hemisphere mid-to-upper tropospheric flow. It improves on existing metrics of ZW1 and ZW3 activity by allowing for variations in both wave phase and amplitude. \citet{Hobbs2010} recently suggested that it is somewhat inappropriate to represent individual zonal wave `events' as constant amplitude ZW1 or ZW3 patterns. Their solution to this problem was to simply consider the two Pacific anticyclones associated with the ZW3 pattern in isolation, however we find that by allowing the amplitude to vary (via the use of a Hilbert Transform) it is both possible and appropriate to consider hemispheric patterns of zonal wave activity.  

The climatology derived from our Planetary Wave Index (PWI) confirms previous results regarding the seasonality of zonal wave activity (peak activity in winter, seasonal migration of the zonal location/phase), and also identifies a large sector of the western hemisphere (120-30$^{\circ}$W) where the mean wave activity breaks down during summer. In contrast to the results presented here, previous studies have suggested a link between stationary wave activity and ENSO \citep[e.g.][]{Trenberth1980,Raphael2003,Hobbs2007}, particularly when in phase with the SAM \citep{Pezza2012}. Given the hemispheric nature of the PWI (i.e. it was designed to respond most strongly to coordinated, hemispheric patterns of meridional flow), it is perhaps not surprising that we found no link with ENSO or SAM. Teleconnections between ENSO and the high southern latitudes tend to be more localised around the southeast Pacific \citep{Turner2004}, while the SAM is not well coordinated between the western and eastern hemispheres \citep{Ding2012}. A detailed look at the relationship between the PWI and variables such as ocean temperature and stratospheric ozone was beyond the scope of this paper, but such an analysis would be warranted in considering the possible predictability of zonal wave variability.

With respect to the link between zonal wave activity and surface extremes, much of the existing literature focuses on sea ice. The latest study from \citet{Raphael2014} takes a new approach to assessing the influence of the atmospheric circulation, focusing on the ice advance (approximately March-August) and retreat (September-February) seasons for five distinct regions of sea ice variability around Antarctica. From looking at the spatial pattern of correlation between sea ice extent and 500 hPa geopotential height for each season/region, they suggest that the ZW3 pattern is the primary driver of sea ice variability in the Weddell and Amundsen/Bellingshausen Seas during the advance season. Our results tend to support this finding, particularly during the early part (MAM) of the advance season. In contrast, the strong association identified between the PWI and sea ice coverage just to the north of George V Land, Ad{\'e}lie Land and the Sabrina Coast in East Antarctica was not supported by \citet{Raphael2014}, who found the SAM to be the major driver in that region for both the advance and retreat seasons. 

For the King Haakon VII Sea (10$^{\circ}$W-70$^{\circ}$E), \citet{Raphael2014} were unable to identify an obvious atmospheric driver. Our results suggest that zonal planetary wave activity may play an important role there, since the correlation patterns identified by \citet{Raphael2014} loosely resemble the mean planetary wave patterns identified in this study. The lack of a closer resemblance may be due to the fact that in some seasons the association between the PWI and sea ice coverage is unidirectional. In MAM, for instance, PWI values greater than the 90th percentile are associated with anomalously low sea ice concentrations, while values less than the 10th percentile are associated with near average (as opposed to anomalously high) concentrations.

For regions of the Southern Hemisphere where orographic precipitation dominates (e.g. New Zealand, Chile, coastal Antarctica), zonal planetary wave activity stands out as an important driver of precipitation variability (\textit{FIXME: Look at articles on rainfall variability in these places and see if zonal waves are mentioned}). It also appears to be associated with wet and cool conditions over eastern and southern Australia during spring. \textit{FIXME: Comment on the existing Australian rainfall variability literature. The overview that everyone cites is \citet{Risbey2009}, which clearly shows that ENSO is the main driver of rainfall variability in spring (perhaps all we're seeing in our precipitation composites is the relationship between the PWI and ENSO, although we found no relationship between those two annually - could check seasonally), while a recent study by \citet{Frederiksen2014} actually mentions ZW3 but mainly with respect to winter rainfall variability in Australia.}

Aside from Australia, the only other Southern Hemisphere locations for which planetary wave activity is associated with large cold temperature anomalies is the Weddell Sea (in autumn and winter) and Ross Sea (spring). In the main, the enhanced meridional flow brings warm air poleward and large positive temperature anomalies are seen throughout most of Antarctica, particularly during autumn and winter. Remarkably, zonal wave activity is scarcely (if ever) mentioned in overviews and analyses of Antarctic temperature variability \citep[e.g.][]{Russell2010, Schneider2011, Yu2011}. Our results suggest that this may be an important oversight, especially given the demonstrated independence of zonal wave activity from SAM and ENSO (which have tended to be the main focus).   

The link between zonal planetary wave activity and West Antarctic temperature variability is particularly interesting given the large positive temperature trends observed in that region over recent decades \citep[e.g.][]{Bromwich2013}. Among a number of other seasons and regions, the winter trends over the interior of West Antarctica \citep{Ding2011} and the spring trends over the western aspect of the Antarctic Peninsula \citep{Ding2013} have been linked to the Pacific-South American (PSA) pattern, which is the most prominent non-zonal planetary wave pattern in the Southern Hemisphere \citep[e.g.][]{Mo2001}. Temperature variability in these seasons/regions was shown here to be strongly associated with the PWI, so future work will attempt to disentangle the influence of the PSA and zonal wave patterns in the region, so as to better understand the role of both in recent trends.   

\textit{FIXME: Address why the new metric/approach is useful. In particular, why would it be useful to be able to identify when the flow is meridional in a coordinated, hemispheric sense, as opposed to using more localized indices like the ASL index \citep{Turner2013}?}

