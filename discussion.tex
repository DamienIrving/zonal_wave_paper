\section{Discussion}

The discussion will contain a recap of the methodology, followed by a discussion of my results as they relate to the literature:
\begin{itemize}
\item General climatological characteristics
\begin{itemize}
\item Alternative view of Hobbs (mention summer breakdown)
\item Not strongly linked to SAM or ENSO
\end{itemize}
\item Sea ice (contrast with Raphael)
\item Rainfall reductions for mountainous regions in Southern Hemisphere
\item Cold and wet in southern/eastern Australia during spring
\item Temperatures elsewhere in Antarctica
\item Future work disentangling PSA and West Antarctic trends
\end{itemize}

A new metric of planetary wave activity has been defined that captures the combined influence of the two major zonal waves in the Southern Hemisphere mid-to-upper tropospheric flow. It improves on existing metrics of ZW1 and ZW3 activity by allowing for variations in both wave phase and amplitude. \citet{Hobbs2010} recently suggested that it is somewhat inappropriate to represent individual zonal wave "events" as constant amplitude ZW1 or ZW3 patterns. Their solution to this problem was to simply consider the two Pacific anticyclones asssociated with the ZW3 pattern in isolation, however we find that by allowing the amplitude to vary (via the use of a Hilbert Transform) it is both possible and appropriate to consider hemispheric patterns of zonal wave activity.  

The climatology dervied from the Planetary Wave Index confirms previous results regarding the seasonality of zonal wave activity (peak activity in winter, seasonal migration in the zonal location/phase), and also indentifies a large sector of the western hemisphere () where the mean wave activity breaks down during summer. FIXME: Comment on variability of the metric and links to ENSO and SAM.









