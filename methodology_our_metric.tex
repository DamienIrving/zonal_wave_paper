\subsection{A new metric: concepts}

Our solution to this problem borrows from recent advances in the automated identification of Rossby wave packets. In particular, \citet{Zimin2003} pioneered a method of identifying the envelope of atmospheric waveforms based on the Hilbert Transform, which is a well known technique in digital signal processing but was previously untried in the atmospheric sciences. The method involves performing a Fourier Transform, followed by an inverse Fourier Transform just for the wavenumbers of interest. The (complex) amplitude of the resulting waveform represents the wave envelope (FIG 1). Subsequent studies have gone on to apply this method in the context of identifying and tracking Rossby wave packets in daily timescale data \citep{Glatt2014,Souders2014a}, however its utility in identifying waveforms on longer temporal and larger spatial spaces has not previously been investigated.

In the context of planetary wave activity in the Southern Hemisphere, the average wave amplitude around a given latitude circle essentailly represents an aggregated measure of the 'waviness' of the flow. A high average value indicates a strong meridional component to the flow around much of the hemisphere (i.e. a coordinated planetary wave pattern), while a low value indicates a strongly zonal flow. By repeating the Hilbert Transform for every latitude cicle, a spatial map of the wave envelope can be constructed, as shown in Figure (FIG2).

