\section{Methodology}\label{s:methodology}

\subsection{Existing metrics of planetary wave activity}\label{s:existing_metrics}

In analyzing SH planetary wave activity (i.e. the ZW1 and/or ZW3 patterns), previous studies have tended to define metrics based on either grid point values or Fourier decomposition. With respect to the former, \citet{Raphael2004} define a ZW3 index that is essentially the average 500 hPa geopotential height zonal anomaly across three key points (the annual average location of the ridges of the ZW3 pattern in the 500 hPa geopotential height field). This metric was applied in a subsequent study of Antarctic sea ice variability \citep{Raphael2007} and is attractive because of its simplicity, however its stationary nature means that it cannot fully capture the subtle (approximately 15 degrees of longitude on average) seasonal migration in the phase of the ZW3 \citep{vanLoon1984,Mo1985} or the occurrence of patterns whose phase does not approximately coincide with the location of the three analysis points.

A number of studies have analyzed the zonal waves by using a Fourier Transform to express the upper tropospheric geopotential height in the frequency domain as opposed to the spatial domain \citep{Hobbs2007,Hobbs2010,Turner2013}. The output of a Fourier Transform can be expressed in terms of a magnitude and phase for each wavenumber (or frequency/harmonic - the terminology differs in the literature), so these studies simply analyzed the magnitude and phase information corresponding to the ZW1 and/or ZW3 pattern. While this might be considered an improvement on the grid point method in the sense that the phase is allowed to vary, a shortcoming is that the result is a constant amplitude wave over the entire longitudinal domain. The two major anticyclones associated with the ZW3 pattern (located near New Zealand and South America respectively) are known to be positively covariant with respect to their location (indicating a coordinated wave pattern) but not amplitude \citep{Hobbs2010}, while in many cases ZW1- and/or ZW3-like variability is only prevalent over part of the hemisphere. As discussed in the seminal work of \citet{vanLoon1972}, it is clear that the other Fourier components (i.e. the non-wavenumber 1 or 3 coefficients) are required to modulate the amplitude of the ZW1 and ZW3 variability, and potentially vital information can be lost if those extra components are not incorporated when defining a metric of planetary wave activity. 

None of the aforementioned studies attempted to combine their ZW1 and ZW3 metrics to get a measure of total planetary wave activity, so for an example of this we must turn to the Northern Hemisphere. In analyzing the relationship between planetary wave activity and regional weather extremes, \citet{Screen2014} calculated the 500 hPa geopotential height Fourier amplitudes for a range of wavenumbers of interest, and then simply counted the number of positive and negative magnitude anomalies. While this may be an appropriate approach for the Northern Hemisphere, it too fails to account for the fact that some of the waveforms in a Fourier Transform simply exist to modulate others (which is something that is noted by \citet{Screen2014}) and thus it may not be appropriate to count all magnitude anomalies.  