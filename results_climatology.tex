\section{Results}

\subsection{Mean characteristics}

Visual inspection of the SH circulation revealed that days (with 30-day running mean applied) of PWI greater than the 90th percentile overwhelming exhibited a mixed ZW1 / ZW3 hemispheric planetary wave pattern (Figure \ref{fig:pwi_spatial_summary}). Days of PWI less than the 90th percentile become increasingly unlikely to exhibit a coordinated wave pattern, so the 90th percentile was taken as a threshold value for planetary wave activity. Consistent with previous studies \citep{vanLoon1984,Mo1985}, the composite mean 500 hPa zonal geopotential height anomaly pattern for days exceeding the 90th percentile migrates zonally by approximately 15$^{\circ}$ from its most easterly location during summer to its most westerly disposition during winter (notwithstanding the fact that the pattern breaks down from around 240-330$^{\circ}$E during summer). It displays little seasonal meridional migration.

The composite mean wave pattern (Figure \ref{fig:pwi_spatial_summary}) has a slightly larger amplitude during the winter months and the frequency of strong planetary wave activity is also far more pronounced at that time of the year (Figure \ref{fig:annual_distribution}). The are no statistically significant linear trends in planetary wave activity for any season, however 1980 was associated with particularly strong wave activity (\textit{FIXME: Look into 1980 in detail}).

