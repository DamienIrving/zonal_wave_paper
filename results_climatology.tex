\section{Results}

\subsection{Mean characteristics}

Visual inspection of the SH circulation revealed that days (with 30-day running mean applied) of PWI greater than the 90th percentile overwhelming exhibited a mixed ZW1 / ZW3 hemispheric planetary wave pattern (Figure \ref{fig:pwi_spatial_summary}). Days of PWI less than the 90th percentile become increasingly unlikely to exhibit a coordinated wave pattern, so the 90th percentile was taken as a threshold value for planetary wave activity. Consistent with previous studies \citep{vanLoon1984,Mo1985}, the composite mean 500 hPa zonal geopotential height anomaly pattern for days exceeding the 90th percentile migrates zonally by approximately 15$^{\circ}$ from its most easterly location during summer to its most westerly disposition during winter (notwithstanding the fact that the pattern breaks down from around 240-330$^{\circ}$E during summer). It displays little seasonal meridional migration.

The composite mean wave envelope (Figure \ref{fig:pwi_spatial_summary}) has a slightly larger amplitude during the winter months and the frequency of strong planetary wave activity is also far more pronounced at that time of the year (Figure \ref{fig:annual_distribution}). In the annual composite (Figure \ref{fig:pwi_spatial_summary}) there are strong wave envelope maxima (i.e. locations where the meridional component of the flow is at its largest) over the Amundsen Sea and to the south of Australia, however it is interesting to note that the Amundsen Sea maxima is absent during summer. By contrast, there is a pronounced maxima to the south of Africa during summer that is not present at other times of the year.

With respect to changes in planetary wave activity over the period 1979-2014, the 1990s were a noticeably quiet period during all seasons (Figure \ref{fig:annual_distribution}, right panel) \textit{(FIXME: What were SAM and ENSO doing during these periods?)}. Activity was much higher before and after that period, with a particularly strong upswing in the 2010s during winter. Given the aforementioned issues regarding the representation of low-frequency variability and trends in reanalysis data, it is difficult to know whether these changes over time are `real,' or simply an artifact of the reanalysis process.
