\section{Results}

\subsection{Mean characteristics}

In order to consider the climatological characteristics of Southern Hemipshere planetary wave activity, the 90th percentile of our metric was used as a threshold value. Composites of the days (with 30-day running mean applied) above that threshold show a pronounced ZW1 and ZW3 signature (Figure \ref{fig:envelope_climatology}). The composite mean wave envelope has a slightly larger amplitude during the winter months, while the frequency of strong planetary wave activity is also far more pronounced during this time of the year (Figure \ref{fig:annual_distribution}).

In the annual composite (Figure \ref{fig:envelope_climatology}) there are strong wave envelope maxima (i.e locations where the meridional component of the flow is at its largest) over the Amundsen Sea and to the south of Australia, however it is interesting to note that the Amundsen Sea maxima is absent during summer. In fact, during summer there is a pronounced maxima to the south of Africa, which is not present at other times of the year.

With respect to changes in planetary wave acitivty over the period 1979-2014, the 1990s were a noticeably quiet period during all seasons (Figure \ref{fig:annual_distribution}, right panel). Activity was much higher before and after that period, with a particuarly strong upswing in the 2010s during winter. Given the aforementioned issues regarding the representation of low-frequency variability and trends in reanalysis data, it is difficult to know whether these changes over time are real, or simply an artefact of the reanalysis process.



