\section{Results}

The results begin with a summary of the spatial and temporal characteristics of SH planetary wave activity, before considering its relationship with both the major modes of SH climate variability (SAM and ENSO) and surface extremes. 

\subsection{Spatial characteristics}\label{s:spatial_characteristics}

Visual inspection of the SH circulation revealed that days (with 30-day running mean applied) of PWI greater than the 90th percentile overwhelming exhibit a mixed ZW1 / ZW3 hemispheric planetary wave pattern (Figure \ref{fig:pwi_spatial_summary}). Days of PWI less than the 90th percentile become increasingly unlikely to exhibit a coordinated hemispheric wave pattern, so the 90th percentile was taken as a threshold value for planetary wave activity. 

Elements of the mixed ZW1 / ZW3 pattern shown in Figure \ref{fig:pwi_spatial_summary} are not unique to days where the PWI is greater than its 90th percentile. As demonstrated in Figure \ref{fig:periodograms} (left panel), the ZW1 component of the flow is relatively insensitive to changes in the strength of the meridional flow. Instead, it appears that the main difference between days of very strong (PWI $>$ 90th percentile) and very weak (PWI $<$ 10th percentile) meridional flow is the prominence of the ZW3 component. While influential at all times, the ZW3 component is far more prominent when the meridional flow is strong and therefore dominates the streamfunction anomaly patterns (and thus surface impacts) discussed in the surface extremes section below. 

Given the dominance of the ZW3, it is not surprising that the ZW3 index of \citet{Raphael2004} shows a reasonably high level of agreement with the PWI (Figure \ref{fig:metric_vs_zw3}). Having said that, it is important to note that the colors of the dots in Figure \ref{fig:metric_vs_zw3} --- which represent the phase of the wavenumber three component of the Fourier transform --- are not randomly distributed. Whenever the phase of the wavenumber three component of the flow does not match up with the location of the three grid points used to calculate the ZW3 index (indicated by the dark red and dark blue colors), a low value is recorded for the ZW3 index. The outlying dots in the bottom right hand quadrant are particularly noteworthy, as in these cases the PWI (and hence in most cases the amplitude of the wavenumber three component of the flow) is actually quite large. The failure of the ZW3 index to capture these out of phase patterns means that composite analyses based on that index may overstate the stationarity (and hence the time-mean impacts) of the ZW3 component of the flow.