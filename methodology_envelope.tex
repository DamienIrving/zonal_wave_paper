\subsection{Wave envelope}

A possible way of addressing the shortcomings of the previous approaches borrows from recent advances in the automated identification of Rossby wave packets. In particular, \citet{Zimin2003} pioneered a method of identifying the envelope of atmospheric waveforms based on the Hilbert transform, which is a well known technique in digital signal processing but had been scarcely applied in the atmospheric sciences. In defining their algorithm, \citet{Zimin2003} consider the real function $\upsilon(x)$ on an equidistant grid along a latitude circle, which is parameterised by $x$, with $0 < x \leq 2\pi$. The grid points are located at $x = 2 \pi l / N$, where $l = 1, 2, \dotsc, N$ and $N$ is an even integer. The first step of the algorithm is to compute the Fourier transform of $\upsilon(x)$:

\begin{equation}\label{eq:fourier_transform}
\hat{\upsilon}_k = \frac{1}{N}\sum_{l=1}^N \upsilon \left( \frac{2 \pi l}{N} \right) e^{-2 \pi ikl/N},\qquad k = -\frac{N}{2} + 1, \dotsc, \frac{N}{2}
\end{equation}

\noindent The inverse Fourier transform is then applied to a selected band ($0 < k_{min} \leq k \leq k_{max}$) of the positive wavenumber half of the Fourier spectrum (this is the part of the process that was inspired by the Hilbert transform):

\begin{equation}\label{eq:inverse_transform}
w \left( \frac{2 \pi l}{N} \right) = 2 \sum_{k=k_{min}}^{k_{max}} \hat{\upsilon}_k e^{2\pi ikl/N}
\end{equation}

\noindent The (complex number) amplitude of the resulting waveform ($w$) represents the wave envelope ($E$):

\begin{equation}\label{eq:wave_envelope}
E(2 \pi l / N) = | w(2 \pi l / N) |
\end{equation}

\noindent The various components of this process are illustrated in Figure \ref{fig:example_hilbert}, whereby $\upsilon(x)$ is the meridional wind along the 54.75$^{\circ}$S latitude circle for two different data times. 

Subsequent studies have gone on to apply the \citet{Zimin2003} algorithm in the context of identifying and tracking Rossby wave packets in daily timescale data \citep{Glatt2014,Souders2014a}, however its utility in identifying waveforms on longer temporal and larger spatial scales has not previously been investigated. In these studies, a spatial map of the wave envelope is constructed for each data time (i.e. $E(t,\lambda,\phi)$, where $t$, $\lambda$ and $\phi$ represent time, latitude and longitude respectively). The utility of these maps is evident when considering the two maps (Figure \ref{fig:example_envelope}) that correspond to the single-latitude examples shown in Figure \ref{fig:example_hilbert}. For (the diurnal averages of) both 22 May 1986 and 29 July 2006, it is clear that the wavenumber three component of the Fourier transform is dominant at 54.75$^{\circ}$S (and at the other nearby latitudes not shown in Figure \ref{fig:example_hilbert}). An analysis based on single wavenumbers could lead one to believe that both data times are associated with a pronounced hemispheric ZW3 pattern, despite the fact that this is clearly only true for 29 July (Figure \ref{fig:example_envelope}). On 22 May the spatial scale of the anomalous flow from 200-260$^{\circ}$E approximately matches wavenumber three, but elsewhere the flow is strongly zonal. The other components of the Fourier transform serve to modulate the wavenumber three component accordingly, and by using the wave envelope as opposed to a single wavenumber approach, this useful information is retained.