\subsection{An alternative ZW3 index?}

The dominace of the ZW1 and ZW3 patterns on Southern Hemisphere planetary wave activity is further emphasised when considering the individual Fourier components calculated during the Hilbert Transform (Figure \ref{fig:fourier_spectrum}). When no temporal smoothing is applied to the daily timescale 500 hPa meridional wind data, wave number four is most dominant. For smoothing of around 10 days and longer wave three is clearly dominant, with wave one becoming progressively more influential as the smoothing increases.

Given that wave three is so dominant at the monthly timescale (i.e. around a 30-day running mean), it is interesting to consider the merits of our planetary wave metric as an alternative metric for the ZW3 pattern itself. This would not work at all timescales (i.e. at a seasonal or 90-day running mean timescale and longer wavenumber one begins to become strongly influential), but the montly timescale represents a sweet spot where almost all large values of our metric manifest as strong ZW3 patterns. 


Show that at 30 day running mean timescale, ZW3 is dominant and could be a good alternative metric, although since they never really occur in isolation, perhaps the current ZW3 index is really looking at total planetary wave activity anyway?