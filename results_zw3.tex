\subsection{An alternative ZW3 index?}

The dominace of the ZW1 and ZW3 patterns on Southern Hemisphere planetary wave activity is further emphasised when considering the individual Fourier components calculated during the Hilbert Transform. Wavenumber three dominates the wavenumber spectrum when the running mean applied to the daily 500 hPa meridional wind data is greater than 10 days, with wavenumber one becoming progressively more influential as the smoothing increases (Figure \ref{fig:fourier_spectrum}).

Given the clear dominance of wavenumber three from around 10 to 90 days of smoothing, it is interesting to consider the merits of the planetary wave index as an alternative metric for the ZW3 pattern on monthly and seasonal timescales. A direct comparison of the planetary wave index and the ZW3 index of \citet{Raphael2004} shows a relatively high level of agreement (as evidenced by the linear regression line in Figure \label{fig:metric_vs_zw3}), however there are some important discrepancies. In particular, the color of the dots in Figure \label{fig:metric_vs_zw3} is not randomly distributed.

Also spreads a great deal into the bottom right quandrant (i.e.  


%mention (perhaps with figure in an appendix) that the spectrum looks much different for geopotential height

%although since they never really occur in isolation, perhaps the current ZW3 index is really looking at total planetary wave activity anyway?