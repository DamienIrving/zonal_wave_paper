\subsection{Hilbert Transform and wave envelope}

Our solution to the shortcomings of previous metrics borrows from recent advances in the automated identification of Rossby wave packets. In particular, \citet{Zimin2003} pioneered a method of identifying the envelope of atmospheric waveforms based on the Hilbert Transform, which is a well known technique in digital signal processing but had been scarcely applied in the atmospheric sciences. The method involves performing a Fourier Transform of the meridional wind along a selected line of latitude ($\phi$), followed by an inverse Fourier Transform just for the wavenumbers of interest. The (complex number) amplitude of the resulting waveform represents the wave envelope $E(\lambda)$, where $\lambda$ denotes longitude (e.g. Figure \ref{fig:example_hilbert}). Subsequent studies have gone on to apply this method in the context of identifying and tracking Rossby wave packets in daily timescale data \citep{Glatt2014,Souders2014a}, however its utility in identifying waveforms on longer temporal and larger spatial scales has not previously been investigated.