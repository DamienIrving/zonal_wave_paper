\subsection{Analysis techniques}

\subsubsection{Anomalies}
All anomaly data discussed in the paper represent the daily anomaly. For instance, in preparing the 30-day running mean surface air temperature anomaly data series, a 30-day running mean was applied to the daily surface air temperature data and then the anomaly at each timestep was calculated by subtracting the mean value for that specific day over the entire study period.

\subsubsection{Composites}
Annual and seasonal mean composites are presented throughout the paper for key variables of interest. To produce these composites, a mean spatial map was calculated across all timesteps that (a) fell within the season of interest, and (b) exceeded a specified threshold value. To determine the statistical significance of the composite mean at each grid point, two-sided, independent sample t-tests were used to calculate the probability ($p$) that the composite mean value was equal to the climatological (i.e. all timesteps) mean for that season.

\subsubsection{Normalisation}
In order to aid visual comparison, a number of plots throughout the paper present the "normalised" values of an index. The normalisation process involves subtracting the mean of the data series and then dividing by the standard deviation. This is more properly known as the standardised anomaly \citep{Wilks2011}. 
%Wilks (p 46) call this a standardised anomaly

\subsubsection{Periodograms} %From Wilks2011
The characterisitcs of data series that have been Fourier-transformed are often summarised using a plot known as a periodogram or Fourier line spectrum \citep{Wilks2011}. These plots are also sometimes referred to as a power or density spectrum, and most commonly display the squared amplitudes $C_k^2$ of the Fourier Transform as a function of their corresponding frequencies $\omega_k$. As an alternative to the squared amplitude, we have chosen to rescale the vertical axis and instead use the $R^2$ statistic commonly computed in regression. The $R^2$ for the $k$th harmonic is simply:

\begin{equation}
R_k^2 = \frac{(n/2)C_k^2}{(n-1)s_y^2}
\end{equation}

where $s_y^2$ is the sample variance of the data series and $n$ the length of the data series. This rescaling is particularly useful as it shows the proportion of variance in the original data series accounted for by that harmonic \citep{Wilks2011}.