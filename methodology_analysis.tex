\subsection{Analysis techniques}

\subsubsection{Composites}
Annual and seasonal mean composites are presented throughout the paper for key variables of interest. To produce these composites, a mean spatial map was calculated across all timesteps that (a) fell within the season of interest, and (b) exceeded a specified threshold value. To determine the statistical significance of the composite mean at each grid point, two-sided, independent sample t-tests were used to calculate the probability ($p$) that the composite mean value was equal to the climatological (i.e. all timesteps) mean for that season.    

\subsubsection{Normalisation}
In order to aid visual comparisons, the "normalised" values of a given index are presented on a number of occasions throughout the paper. The normalisation process involves subtracting the mean of the data series and then dividing by the standard deviation. 
%Wilks (p 46) call this a standardised anomaly

FIXME: Explain that anomalies are the daily anomaly

FIXME: Explain spectral density analysis