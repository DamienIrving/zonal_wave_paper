\subsection{A new metric: details}

To define and calculate a metric of zonal planetary wave activity, a spatial map of the wave envelope was obtained for each timestep by performing a Hilbert Transform along each latitude circle. The 500 hPa meridional wind was used for the Transform, and wavenumbers 1-9 were retained. The latitudinal dimension of each map was eliminated by calculating the meridional maximum over the range 40$^{\circ}$-70$^{\circ}$S, and then the median was taken in order to arrive at a single value. A number of factors were taken into consideration in selecting this methdology:
\begin{itemize}
\item The geopotential height (or its zonal anomaly) could have been used instead of the meridional wind, however many other authors think the meridional wind is great \citep[e.g.]{Hope2014}. EXPLAIN WHY.
\item The atmospheric level is somewhat unimportant since planetary waves are equivalent barotropic, so 500 hPa simply represents a mid-to-upper tropospheric level that is below the tropopause in all seasons and at all latitudes of interest.
\item The wave envelope is slightly smoother if wavenumbers greater than 9 are left out of the Hilbert Transform, but otherwise the result is not appreciably different from when all wavenumbers are retained.  
\item The median (as opposed to the mean or integral) was taken to guard against large values in one part of the hemisphere overly influencing the end result.
\end{itemize}

In order to be consistent with most of the existing literature, the majority of the analysis focuses on monthly timescale data. Instead of using monthly data, we used a 30-day running mean. I THINK I HAVE SOME NOTES ON WHY THIS IS GOOD.

