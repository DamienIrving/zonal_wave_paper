\subsection{A new metric: details}

To define and calculate a metric of zonal planetary wave activity, a spatial map of the wave envelope was obtained for each timestep by performing a Hilbert Transform along each latitude circle. The 500 hPa meridional wind was used for the Transform, and wavenumbers 1-9 were retained. The latitudinal dimension was then eliminated by calculating the meridional maximum over the range 40$^{\circ}$-70$^{\circ}$S, and then the median was taken in order to arrive at a single value. A number of factors were taken into consideration in selecting this methdology:
\begin{itemize}
\item The geopotential height (or its zonal anomaly) would have been an alternative option, however many other authors think the meridional wind is great \citep[e.g.]{Hope2014}. EXPLAIN WHY.
\item The precise atmospheric level is somewhat unimportant since planetary waves are equivalent barotropic, so 500 hPa simply represents a mid-to-upper tropospheric level that is below the tropopause in all seasons and at all latitudes of interest.
\item Wavenumbers 1-9 were retained in the Transform. At 500 hPa this is effectively all wavenumbers, however the resulting wave envelope is slightly smoother if wavenumbers greater than 9 are left out.
\end{itemize}

In order to be consistent with most of the existing literature, the majority of the analysis focuses on monthly timescale data. Instead of using monthly data, we used a 30-day running mean. I THINK I HAVE SOME NOTES ON WHY THIS IS GOOD.

