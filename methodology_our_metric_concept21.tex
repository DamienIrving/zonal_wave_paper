In the context of planetary wave activity in the Southern Hemisphere, the average (or median or integral) wave amplitude around a given latitude circle essentailly represents an aggregated measure of the `waviness' of the hemispheric-scale flow. A high average value indicates a strong meridional component to the flow around much of the hemisphere (i.e. a coordinated planetary wave pattern), while a low value indicates a strongly zonal flow. 

By repeating the Hilbert Transform for every latitude circle, a spatial map of the magnitude of the wave envelope can be constructed for each timestep. The utility of these maps is evident when considering the two maps (Figure \ref{fig:example_envelope}) that correspond to the single-latitude Hilbert Transforms shown in Figure \ref{fig:example_hilbert}. For both 22 May 1986 and 29 July 2006, it is evident that the wavenumber three component of the Fourier Transform is dominant at 55$^{\circ}$S (and at the other nearby laitudes not shown in Figure \ref{fig:example_hilbert}). An analysis based on single wavenumbers may interpret this to mean both timesteps are associated with a pronounced hemispheric ZW3 pattern, despite the fact that this is clearly only true for 29 July (Figure \ref{fig:example_envelope}). On 22 May the spatial scale of the anomlous flow from 200-260$^{\circ}$E approximately matches wavenumber three, but elsewhere the flow is strongly zonal. The other components of the Fourier Transform on this date exist to modulate the wavenumber three component accordingly, and by using the Hibert Tranform as opposed to a single wavenumber approach, this useful information is retained.
