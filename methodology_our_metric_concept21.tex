In the context of planetary wave activity in the Southern Hemisphere, the average wave amplitude around a given latitude circle (or the median or integral) essentailly represents an aggregated measure of the `waviness' of the hemispheric-scale flow. A high average value indicates a strong meridional component to the flow around much of the hemisphere (i.e. a coordinated planetary wave pattern), while a low value indicates a strongly zonal flow. 

By repeating the Hilbert Transform for every latitude cicle, a spatial map of the wave envelope can be constructed for each timestep (e.g. Figure \ref{fig:example_envelope}). The utility of these maps is evident when considering the example maps in Figure \ref{fig:example_envelope}, in conjunction with the corresponding single-latitude Hilbert Transforms in Figure \ref{fig:example_hilbert}. For both 22 May 1986 and 29 July 2006, it is evident that the wavenumber three component of the Fourier Transform was dominant at 55$^{\circ}$S (and at the other nearby laitudes not shown in Figure \ref{fig:example_hilbert}). An analysis based on single wavenumbers may interpret this to mean both timesteps were associated with a pronounced hemispheric ZW3 pattern, despite the fact that it is clear from Figure \ref{fig:example_envelope} that this is only true for 29 July. On May 22 the spatial scale of the anomlous flow from 200-260$^{\circ}$E is approximately wavenumber three, but elsewhere the flow is strongly zonal. The other components of the Fourier Transform on this date clearly exist to modulate the wave three component accordingly, and by using the Hibert Tranform as opposed to a single wavenumber approach, this useful information is retained.
