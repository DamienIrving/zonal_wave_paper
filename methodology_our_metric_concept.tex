\subsection{A new metric: concept}

Our solution to this problem borrows from recent advances in the automated identification of Rossby wave packets. In particular, \citet{Zimin2003} pioneered a method of identifying the envelope of atmospheric waveforms based on the Hilbert Transform, which is a well known technique in digital signal processing but was previously untried in the atmospheric sciences. The method involves performing a Fourier Transform, followed by an inverse Fourier Transform just for the wavenumbers of interest. The (complex number) amplitude of the resulting waveform represents the wave envelope (FIG 1). Subsequent studies have gone on to apply this method in the context of identifying and tracking Rossby wave packets in daily timescale data \citep{Glatt2014,Souders2014a}, however its utility in identifying waveforms on longer temporal and larger spatial spaces has not previously been investigated.

In the context of planetary wave activity in the Southern Hemisphere, the average wave amplitude around a given latitude circle (or the median or integral) essentailly represents an aggregated measure of the 'waviness' of the flow. A high average value indicates a strong meridional component to the flow around much of the hemisphere (i.e. a coordinated planetary wave pattern), while a low value indicates a strongly zonal flow. 

By repeating the Hilbert Transform for every latitude cicle, a spatial map of the wave envelope can be constructed for each timestep (e.g. FIG 2). The utility of these maps is evident when considering the example maps in FIG2 in conjunction with the corresponding single-latitude Hilbert Transforms in FIG1. For both 22 May 1986 and 29 July 2006, it's evident from Figure FIG1 that the wavenumber three component of the Fourier Transform was dominant at 55S (and the other middle laitudes not shown in Figure FIG1). An analysis based on single wavenumbers may interpret this to mean both timesteps were associated with a pronounced hemispheric ZW3 pattern, despite the fact that it is clear from Figure FIG2 that this is only true for 29 July. By using a Hilbert Transform and retaining all wavenumbers during the Inverse Fourier Transform
