\subsubsection{Precipitation}

The largest composite mean precipitation anomalies were associated with either enhanced or suppressed flow over significant orography (Figure \ref{fig:pr_composite}). For instance, the same anomalous onshore flow that was associated with warm temperatures over both West Antarctica and Wilkes Land was also associated with large positive coastal precipitation anomalies, with the precise location of these anomalies moving with seasonal variations in the location of the mean planetary wave pattern. In contrast, weakened westerly flow over the southern Andes during autumn, winter and spring (but not summer due to the breakdown of the wave pattern in that region) was associated with large negative precipitation anomalies over Chilean Patagonia. A similar mechanism explains the large negative anomalies over New Zealand during winter and spring, when the mean planetary wave pattern is located far enough to the west to exert an appreciable influence on the westerly flow over the South Island. Enhanced orographic precipitation due to anomalous onshore flow might also play a role in the large positive precipitation anomalies seen over eastern Australia during spring, however the anomalies extend far beyond the Great Dividing Range, suggesting that enhanced moisture transport from the Tasman Sea might be the dominant mechanism. 
