\label{fig:metric_vs_zw3}
PWI versus the ZW3 index of \citet{Raphael2004}. Both were calculated using 500 hPa, 30-day running mean data (the PWI was calculated from the meridional wind and the ZW3 index from the geopotential height zonal anomaly). Colors represent the phase of the wavenumber three component of the Fourier transform of the meridional wind (expressed as the location, in degrees east, of the first local maxima), while the red line is a linear least-squares line of best fit. Both indices have been normalized to aid visual comparison (for each index this involved subtracting the mean of the index series and then dividing by the standard deviation).
    