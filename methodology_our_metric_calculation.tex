\subsection{A new metric of planetary wave activity: calculation}

To define and calculate an appropriate metric of zonal planetary wave activity (which we will refer to as the Planetary Wave Index; PWI), a spatial map of the wave envelope was obtained for each timestep by performing a Hilbert Transform along each latitude circle. The 500 hPa meridional wind was used for the Transform, and wavenumbers 1-9 were retained. The latitudinal dimension of each map was eliminated by calculating the meridional maximum over the range 40-70$^{\circ}$S, and then the zonal median was taken to eliminate the longitudinal dimension and arrive at a single value. 

The meridional wind ($v$) was used because it fundamentally reflects the presence of waves in the zonal flow. If the flow is purely zonal there are no waves and $v = 0$, while the magnitude of $v$ reflects the activity of the waves. The meridional wind is also directly involved with meridional changes (e.g. of energy) that are central to the workings of the atmospheric circulation and energy budget, and many studies have shown that $v$, either filtered or unfiltered, contains much dynamic information about synoptic processes (e.g., Trenberth 1991; Berbery and Vera 1996; Yin and Battisti 2004; Hoskins and Hodges 2005; Carmo and de Souza 2009; Petoukhov et al. 2013). 

Another reason for dealing with $v$ as opposed to the geopotential height ($z$) is that the latter tends to be dominated by the long waves. One can see this from the geostrophic relation $v = dZ / dx$. It follows that in Fourier space $v_k = k Z_k$ for any given wavenumber $k$, meaning more of the variance in $v$ is explained by the synoptic (and shorter) waves than it is for $Z$.

Besides the selection of the meridional wind, a number of other factors were taken into consideration in selecting this methodology:
\begin{itemize}
\item The atmospheric level is relatively unimportant since planetary waves are equivalent barotropic. 500 hPa simply represents a mid-to-upper tropospheric level that is below the tropopause in all seasons and at all latitudes of interest.
\item The wave envelope is slightly smoother if wavenumbers greater than 9 are left out of the Hilbert Transform, but otherwise the result is not appreciably different from when all wavenumbers are retained.
\item The meridional maximum (over 40-70$^{\circ}$S) was taken to allow for slight north/south variations in the mean latitude of planetary wave activity and also for the fact that the waveform is not perfectly zonally oriented. 
\item The zonal median (as opposed to the mean or integral) was taken to guard against large values in one part of the hemisphere overly influencing the end result.
\end{itemize}

In order to be consistent with much of the existing literature, the majority of the analysis focuses on the monthly timescale. Monthly mean data were obtained by applying a 30-day running mean to the daily ERA-Interim data, so as to maximize the monthly information available from the dataset. As noted by previous authors \citep[e.g.][]{Kidson1988}, potentially useful information may be lost if only twelve (i.e. calendar month) samples are taken every year. Dates are labeled according to the central date of the 30-day period.   