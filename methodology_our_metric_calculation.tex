\subsection{A new metric of planetary wave activity: calculation}

To define and calculate a metric of zonal planetary wave activity (which we will refer to as the Planetary Wave Index; PWI), a spatial map of the wave envelope was obtained for each timestep by performing a Hilbert Transform along each latitude circle. The 500 hPa meridional wind was used for the Transform, and wavenumbers 1-9 were retained. The latitudinal dimension of each map was eliminated by calculating the meridional maximum over the range 40-70$^{\circ}$S, and then the zonal median was taken to eliminate the longitudinal dimension and arrive at a single value. A number of factors were taken into consideration in selecting this methdology:
\begin{itemize}
\item The geopotential height (or its zonal anomaly) could have been used instead of the meridional wind, but wind is better because... \textit{HELP IAN: How would I best explain that the wind is a better choice? We had dicussed \citet{Hope2014} as a possible reference but they really only give some references about why the meridional wind is good for locating fronts and synoptic disturbances.}
\item The atmospheric level is somewhat unimportant since planetary waves are equivalent barotropic, so 500 hPa simply represents a mid-to-upper tropospheric level that is below the tropopause in all seasons and at all latitudes of interest.
\item The wave envelope is slightly smoother if wavenumbers greater than 9 are left out of the Hilbert Transform, but otherwise the result is not appreciably different from when all wavenumbers are retained.
\item The meridional maxiumum (over 40-70$^{\circ}$S) was taken to allow for north/south variations in the mean latitude of planetary wave activity and also for the fact that in most cases the waveform is not perfectly zonally oriented. 
\item The zonal median (as opposed to the mean or integral) was taken to guard against large values in one part of the hemisphere overly influencing the end result.
\end{itemize}

In order to be consistent with much of the existing literature, the majority of the analysis focuses on the monthly timescale. Monthly mean data were obtained by applying a 30-day running mean to the daily timescale ERA-Interim data, so as to maximise the monthly timescale information available from the dataset. As noted by previous authors \citep[e.g.][]{Kidson1988}, potentially useful information may be lost if only twelve calendar month samples are taken every year. Dates are labelled according to the central date of the 30-day period.   