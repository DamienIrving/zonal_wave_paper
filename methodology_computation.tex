\section{Computation procedures}\label{s:computation}

The results in this paper were obtained using a number of different software packages. A collection of command line utilities known as the NetCDF Operators (NCO) and Climate Data Operators (CDO) were used to edit the attributes of netCDF files and to perform routine calculations on those files (e.g. the calculation of anomalies and climatologies) respectively. For more complex analysis and visualization, a Python distribution called Anaconda was used. In addition to the Numerical Python \citep[NumPy;][]{VanDerWalt2011} and Scientific Python (SciPy) libraries that come installed by default with Anaconda, a collection of Python libraries known as the Climate Data Analysis Tools \citep[CDAT;][]{Doutriaux2009} were used for data analysis. Similarly, in addition to Matplotlib \citep[the default Python plotting library;][]{Hunter2007}, Iris and Cartopy were also used in generating many of the figures.

To facilitate the reproducibility of the results presented, an accompanying Figshare repository has been created to thoroughly document the computational methodology \citep{Irving2015}. In addition to a more detailed account (i.e. version numbers, web addresses) of the software discussed above, that repository contains a supplementary file for each figure in the paper, outlining the computational steps performed in its creation from initial download of the ERA-Interim data through to the final generation of the plot. A version controlled repository of the code referred to in those supplementary files can be found at \url{https://github.com/DamienIrving/climate-analysis}.

    