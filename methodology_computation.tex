\section{Computation}\label{s:computation}

The results in this paper were obtained using a number of different software packages. A collection of command line utilities known as the NetCDF Operators (NCO) and Climate Data Operators (CDO) were used to edit the attributes of netCDF files and to perform routine calculations on those files (e.g. the calculation of anomalies and climatologies) respectively. For more complex analysis and visualisation, a Python distribution called Anaconda was used. In addition to the Numerical Python \citep[NumPy;][]{VanDerWalt2011} and Scientific Python (SciPy) libraries that come installed by default with Anaconda, a collection of Python libraries known as the Climate Data Analysis Tools \citep[CDAT;][]{Doutriaux2009} were used for data analysis. Similarly, in addition to Matplotlib \citep[the default Python plotting library;][]{Hunter2007}, Iris and Cartopy were also used in generating many of the figures. These software packages and libraries were installed and run on an Ubuntu operating system. The precise details of the operating system and various software packages are as follows:
\begin{itemize}
\item Ubuntu. 12.04. April 2012. Canonical Ltd. http://www.ubuntu.com/
\item netCDF Operators. 4.0.8. April 2011. netCDF Operators Project. http://sourceforge.net/projects/nco/
\item Climate Data Operators. 1.5.3. October 2011. Max Plank Institut f{\"u}r Meteorologie. Hamburg, Germany. https://code.zmaw.de/projects/cdo
\item Python. 2.7.8. July 2014. Python Software Foundation. https://www.python.org/
\item Anaconda. 2.0.1. July 2014. Continuum Analytics. Austin, Texas. http://docs.continuum.io/anaconda/
\item NumPy. 1.9.0. September 2014. NumPy Developers. http://www.numpy.org/
\item SciPy. 0.14.0. May 2014. SciPy Developers. http://www.scipy.org/scipylib/index.html
\item cdat-lite. 6.0rc2. June 2011. Program For Climate Model Diagnosis and Intercomparison. Lawrence Livermore National Laboratory, Livermore, California. https://pypi.python.org/pypi/cdat-lite
\item Matplotlib. 1.4.0. August 2014. Matplotlib Development Team. http://matplotlib.org/
\item Iris. 1.7.2. October 2014. Met Office. Exeter, England. http://scitools.org.uk/
\item Cartopy. 0.11. June 2014. Met Office. Exeter, England. http://scitools.org.uk/
\end{itemize}

A supplementary file has been provided for each figure in this paper, outlining the computational steps performed in its creation (i.e. from initial download of the ERA-Interim data through to the final generation of the plot - see the folder tab at the top left of the screen). A version controlled repository of the code referred to in those supplementary files can be found at https://github.com/DamienIrving/climate-analysis, while a static snapshot of the repository at the time of publication can be found HERE (\textit{FIXME: Generate snapshot and add link as soon as paper is accepted}). The provision of this supplementary information and code is an attempt to enhance the reproducibility of the results presented. See \citet{Irving2015} for a detailed discussion of the rationale behind this approach and how it could one day represent a minimum expectation of authors in the weather and climate sciences. 