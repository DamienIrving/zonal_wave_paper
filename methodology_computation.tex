\section{Methodology}\label{s:methodology}

\subsection{Computational details}\label{s:computational_details}

The results in this paper were obtained using Python 2.7.8 and a collection of command line utilities known as the Climate Data Operators \citep[CDO;][]{Schulzweida2014} and netCDF operators \citep[NCO;][]{Zender2014}. 


Supplementary files are provided for each figure, which outline the computational steps performed in their creation. The details of the code, software, and system configuration used in those computational steps is as follows:   
\begin{itemize}
\item Operating system: Ubuntu 12.04, 64-bit %vortex
\item Software: 
\begin{itemize}
\item Climate Data Operators, version 1.5.3 \citep{Schulzweida2014} % in file attributes
\item netCDF Operators, version 4.0.8 \citep{Zender2014} % ncrename --version
\item Python 2.7.8, Anaconda 2.0.1 (64-bit), (default, Aug 21 2014, 18:22:21). Plus following added/non-standard libraries:  % Header info when you run IPython
\begin{itemize}
\item windspharm 1.3.1 %from windspharm.__version__
\item cdat-lite 6.0rc2 %from binstar description
\item iris \citep{Iris}
\end{itemize}
\end{itemize}
\item Code repository: GitHub URL (i.e. so people can get the latest version)
\item License: GNU General Public License (Version 3) (http://www.gnu.org/licenses) %This is what people tend to use in Journal of Open Research Software
\end{itemize}



The current irreproducibility crisis  of the computational results reported 
Each figure presented in this paper comes with a corresponding supplementary file describing the computational steps performed in its creation. The details of the code, software, and system configuration used in those computational steps is as


In an attempt to enhance the reproducibility of the computational results presented in this paper, a series of supplementary files accompany this paper, which each describe the computational steps performed in the create of each figure.    