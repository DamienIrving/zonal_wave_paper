\section{Methodology}\label{s:methodology}

\subsection{Computational details}\label{s:computational_details}

The results in this paper were obtained using a Python distribution called Anaconda and a collection of command line utilities known as the Climate Data Operators (CDO) and netCDF operators (NCO). In addition to the Scientific Python \citep[SciPy;][]{SciPy} software stack that comes installed by default with Anaconda, additional Python libraries known as Climate Data Analysis Tools \citep[CDAT;][]{Doutriaux2009}, iris and cartopy were also used. These software were all installed and run on an Ubuntu operating system. The precise details of the operating system and various software packages (inlcuding name, version number, release date, institution and DOI or URL) are as follows:
\begin{itemize}
\item Ubuntu. 12.04. April 2012. Canonical Ltd. http://www.ubuntu.com/
\item Python. 2.7.8. July 2014. Python Software Foundation. https://www.python.org/
\item Anaconda. 2.0.1. July 2014. Continuum Analytics. Austin, Texas. http://docs.continuum.io/anaconda/
\item Climate Data Operators. 1.5.3. October 2011. Max Plank Institut f{\"u}r Meteorologie. Hamburg, Germany. https://code.zmaw.de/projects/cdo
\item netCDF Operators. 4.0.8. April 2011. netCDF Operators Project. http://sourceforge.net/projects/nco/
\item cdat-lite. 6.0rc2. June 2011. Program For Climate Model Diagnosis and Intercomparison. Lawrence Livermore National Laboratory, Livermore, California. https://pypi.python.org/pypi/cdat-lite
\item Iris. 1.7.2. October 2014. Met Office. Exeter, England. http://scitools.org.uk/
\item Cartopy. 0.11. June 2014. Met Office. Exeter, England. http://scitools.org.uk/
\end{itemize}

A supplementary file has been provided for each figure in this paper, outlining the computational steps performed in its creation (i.e. from initial download of the ERA-Interim data through to the final generation of the plot). The code referred to in those supplementary files can be found HERE (GENERATE DOI FOR MY BITBUCKET REPO). The provision of this supplementary information and code is an attempt to enhance the reproducibility of the results presented. See \citet{Irving2015} for a detailed discussion of the rationale behind this approach and how it could one day represent a minimum expectation of authors in the weather and climate sciences. 