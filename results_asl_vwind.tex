\subsubsection{Amundsen Sea Low}

\textit{ADDME? Analysis of the Amundsen Sea Low Index \citep{Turner2013} shows that the ASL is slightly deeper when the PWI > 90th percentile (975.6 hPa vs. 976.2 hPa, $p = 0.003$) and located farther to the west (224.9$^{\circ}$E vs. 233.4$^{\circ}$E, $p = 8.4e^{-17}$). There's no difference in latitude.  In need to check the seasonal results, because the ASL is deepest and farthest west during winter, which is when planetary wave activity is most common.}  

\subsubsection{Disorganized meridional flow}

The strong, coherent circulation patterns seen in composites of days where the PWI is greater than its 90th percentile suggest that essentially all days with a strong meridional component to the 500hPa hemispheric flow form a coordinated (or organized) hemispheric wave pattern. In order to confirm this suggestion, the average amplitude of the meridional wind over 40-70$^{\circ}$S (which is a simple metric with no preference for organized or disorganized flow) was correlated with the PWI. The very high level of agreement ($r = 0.93$) confirms that there is a strong tendency for days of large hemispheric meridional flow to organize into a mixed ZW1 / ZW3 pattern. 

A corollary to this finding is that while the wave envelope field $E(\lambda,\phi)$ provided valuable spatial information about the characteristics of SH planetary wave activity (e.g. Figure \ref{fig:envelope_climatology}), its subsequent application in defining the PWI turned out to be unnecessarily complex in this special case. Unlike the simple average amplitude metric, it should be said that the general-purpose characteristics of the PWI mean that it could also be applied in analyzing restricted longitudinal domains and/or regions where the mean meridional wind is not close to zero, which is a focus of future work.
