\subsubsection{Amundsen Sea Low}

\textit{ADDME? Analysis of the Amundsen Sea Low Index \citep{Turner2013} shows that the ASL is slightly deeper when the PWI > 90th percentile (975.6 hPa vs. 976.2 hPa, $p = 0.003$) and located farther to the west (224.9$^{\circ}$E vs. 233.4$^{\circ}$E, $p = 8.4e^{-17}$). There's no difference in latitude.  In need to check the seasonal results, because the ASL is deepest and farthest west during winter, which is when planetary wave activity is most common.}  

\subsubsection{Disorganized meridional flow}

By taking the meridional maximum of the wave envelope field $E(\lambda,\phi)$ followed by the zonal median, the PWI was designed to (a) capture slightly non-zonal waves, and (b) bias against flow regimes that are strongly meridional in only part of the hemisphere. While these are valid design choices, the strong, coherent circulation patterns seen in composites of days where the PWI is greater than its 90th percentile suggest that essentially all days with strong meridional component to the 500hPa hemispheric flow form a coordinated (or organized) hemispheric wave pattern. In other words, while wave envelope field $E(\lambda,\phi)$ provides valuable spatial information about the characteristics of SH planetary wave activity (e.g. Figure \ref{fig:envelope_climatology}), its subsequent application in defining the PWI might have been unnecessarily complex in this special case. In order to confirm this suggestion, the average amplitude of the meridional wind over 40-70$^{\circ}$S (which is a simpler metric with no preference for organized or disorganized flow) region was correlated with the PWI. The very high level of agreement ($r = ??$) confirms that there is a strong tendency for days of large hemispheric meridional flow to organize into a mixed ZW1 / ZW3 pattern. The PWI would be useful when analyzing restricted longitudinal domains and/or regions where the mean meridional wind is not close to zero, niether of which was true in this case.