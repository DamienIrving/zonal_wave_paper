\subsubsection{Disorganized meridional flow}

The strong, coherent circulation patterns seen in composites of days where the PWI is greater than its 90th percentile suggest that the vast majority of days with a strong meridional component to the 500hPa hemispheric flow form a coordinated (or organized) hemispheric wave pattern. In order to confirm this suggestion, the average amplitude of the meridional wind ($\bar{\lvert v \rvert}$) over 40-70$^{\circ}$S; which is a simple metric with no preference for organized or disorganized flow) was correlated with the PWI. The very high level of agreement ($r = 0.93$) confirms that there is a strong tendency for days of large hemispheric meridional flow to organize into a mixed ZW1 / ZW3 pattern. 

A corollary to this finding is that while the wave envelope field $E(\lambda,\phi)$ provided valuable spatial information about the characteristics of SH planetary wave activity (e.g. Figure \ref{fig:envelope_climatology}), an alternative to using it to define the PWI would have been to simply use $\bar{\lvert v \rvert}$ as our metric of planetary wave activity. This alternative is only possible because of the special nature of the analysis; $\bar{v} \approx 0$ at all timesteps and essentially all wavenumbers were retained in calculating $E(t,\lambda,\phi)$. Future work will look at restricted longitudinal domains (where $\bar{v} \neq 0$) and wavenumber ranges, which is where the utility of $E(t,\lambda,\phi)$ can be exploited to construct useful indices similar to the PWI. 
