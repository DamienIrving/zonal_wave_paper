\subsection{Surface conditions}\label{s:surface_conditions}

In order to assess the influence of planetary wave activity on regional climate variability, the composite means of the variables of interest (surface air temperature anomaly, precipitation anomaly and sea ice concentration anomaly) were calculated for all data times where the PWI exceeded its 90th percentile. In other words, we asked the question: what is the average temperature (or precipitation or sea ice concentration) anomaly when there is strong planetary wave activity? The anomalous flow associated with these composites (indicated by the 500 hPa streamfunction anomaly as opposed to the streamfunction \textit{zonal} anomaly shown earlier) has a very strong ZW3 signature. This is consistent with the spatial characteristics presented earlier, which indicate that the distinguishing feature of days of strong meridional flow is the enhanced ZW3 component (as opposed to ZW1).  

\subsubsection{Surface air temperature}

Planetary wave activity was found to be associated with large and widespread surface air temperature anomalies over and/or around much of West Antarctica during all seasons (Figure \ref{fig:tas_composite}). The most pronounced anomalies were seen during autumn and winter, with warmer than average conditions over the interior of West Antarctica (associated with an anomalous northerly flow) and correspondingly colder than average conditions over the Weddell Sea (associated with an anomalous southerly flow). Due to the aforementioned seasonal migration (and breakdown during summer) of the mean planetary wave pattern, warm anomalies were confined to the Antarctica Peninsula during spring, while summer was associated with the smallest anomalies of any season.  

With respect to other sectors of the high southern latitudes, anomalously warm temperatures were widespread over East Antarctica during all seasons except summer. The largest anomalies were seen over Wilkes Land during autumn and winter, in association with an anomalous northerly flow in that region. Other features of note included anomalously cool temperatures over the Ross Sea and mainland Australia during spring.

    
    