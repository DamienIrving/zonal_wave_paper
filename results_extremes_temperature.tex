\subsection{Surface extremes}

Two approaches were taken to assessing the influence of planetary wave activity on surface extremes. For the first, the composite mean of a variable of interest (surface air temperture anomaly, precipitation anomaly or sea ice fraction anomaly) was calculated for all timesteps where the PWI exceeded the 90th percentile. In other words, we first asked the question: what is the average temperature (or precipitation or sea ice fraction) anomaly when there is strong planetary wave activity? To check the robustness of these results the reverse approach was also applied, whereby the average PWI was calculated at each grid point for all timesteps where the variable of interest exceeded its 90th percentile. In other words: what is the average value of the PWI when ... FIXME

FIXME: ALso look at composites of the lowest 10 percent of days. Might not need to show results, but useful to comment that the lowest 10 percent aren't reverse of highest 10 percent.

\subsubsection{Surface air temperature}

MAM and JJA heat throughout all of Antarctica. Particularly hot in West Antarctica in MAM and JJA and over the Antarctic Peninsula in SON (FIXME: talk about possible mechanisms). Also cold in southern Australia during SON that strong southerly flow.


%mention/check that the result is no different using linearly detrended temperature data