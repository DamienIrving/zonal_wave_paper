\subsection{Surface extremes}

In order to assess the influence of planetary wave activity on surface extremes, the composite mean of a variable of interest (surface air temperture anomaly, precipitation anomaly or sea ice fraction anomaly) was calculated for all timesteps where the PWI exceeded the 90th percentile. In other words, we asked the question: what is the average temperature (or precipitation or sea ice fraction) anomaly when there is strong planetary wave activity? To check the robustness of these results the reverse approach was also applied, whereby the average PWI was calculated at each grid point for all timesteps where the variable of interest exceeded its 90th percentile. In other words: what is the average value of the PWI when the temperature (or precipitation or sea ice fraction) is extreme? In all cases this reverse approach simply confirmed the existing results (i.e. no major new insights were gained) so the results are not discussed (FIXME: double check that this is true).

While the impact of a strongly zonal hemispheric flow is not the focus of this paper, composites for all timesteps where the PWI was less than the 10th percentile were also created (Appendix \ref{s:zonal_composites}). It is important to note that these composite mean fields for surface temperature, precipitation and sea ice fraction were not simply the opposite sign to the 90th percentile fields. In other words, in many regions a strongly zonal hemispheric flow is not associated with surface impacts opposite to that of high planetary wave activity. This is not necessarily surprising, considering that the sample of days when a coordinated, hemispheric-wide planetary wave pattern is present (i.e. days of high PWI) is only a subset of the total sample of days for which there is a strong meridional component to the hemispheric flow.

\subsubsection{Surface air temperature}

Planetary wave activity was found be associated with large and widespread surface air temperature anomalies over and/or around much of West Antarctica during all seasons (Figure \ref{fig:tas_composite}). The most pronounced anomalies were seen during autumn and winter, with much warmer than average conditions over the interior of West Antarcica (due to the anomalous northerly flow) and correspondly colder than average conditions over the Weddell Sea (due to the anomalous southerly flow). Due to the aforementioned seasonal migration (and breakdown during summer) of the mean planetary wave pattern, during spring warm anomalies were confined to the Antarctica Peninsula, while summer was associated with the smallest anomalies of any season.  

With respect to the remainder of the Southern Hemisphere, anomalously warm temperatures were widespread over East Antarctica during all seasons except summer. The largest anomalies were seen over Wilkes Land during autumn and winter, in association with the anomalous northerly flow in that region. Other features of note included anomalously cool temperatures over the Ross Sea and mainland Australia during spring.

%mention/check that the result is no different using linearly detrended temperature data