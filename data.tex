\section{Data}\label{s:data}

The series of reliable, spatially complete atmospheric data available for the mid-to-high southern latitudes is relatively short. The reanalysis projects have produced sequences of surface and upper air fields that in some cases date back to the 1950s \citep{Kistler2001,Uppala2005,Kobayashi2015}, however it is generally accepted that these have limited value prior to 1979 at high southern latitudes, due to a lack of satellite sounder data for use in the assimilation process \citep{Hines2000}.

The latest generation reanalysis datasets (which all date back to at least 1979) are the European Centre for Medium-Range Weather Forecasts Interim Reanalysis dataset \citep[ERA-Interim;][]{Dee2011}, Modern Era Retrospective-analysis for Research and Applications \citep[Merra;][]{Rienecker2011}, Climate Forecast System Reanalysis \citep[CFSR;][]{Saha2010} and Japanese 55-year reanalysis \citep[JRA-55;][]{Kobayashi2015}. While assessments of the validity of these datasets in the mid-to-high southern latitudes have only just begun to emerge, the limited evidence suggests that ERA-Interim may be the superior product. In comparison to its peers, ERA-Interim best reproduces the mean sea level pressure and 500-hPa geopotential height at station locations around Antarctica \citep{Bracegirdle2012}, and provides the most realistic depiction of precipitation variability \citep{Bromwich2011,Nicolas2011}. As such, daily timescale ERA-Interim data (for the period 1 January 1979 to 31 December 2014) was used in this study.

While ERA-Interim may be (tentatively) considered the superior reanalysis product, all reanalysis datasets need to be treated with caution in the mid-to-high Southern latitudes due to the sparsity of observational data. There are also well-known difficulties with the representation of low-frequency variability and trends in reanalysis data. Spurious shifts and other artifacts can be present due to changes in the observing system, transitions between multiple production streams, or various other mistakes that can occur in a complex reanalysis production \citep{Dee2014}. Given these issues, it is probably not surprising that surface temperature trends over Antarctica, for instance, vary greatly amongst the latest reanalysis datasets \citep{Nicolas2014}. These issues are somewhat ameliorated by the fact that the primary focus of this study is seasonal and interannual variability (as opposed to long-term trends or low-frequency variability), but they are still important to keep in mind.